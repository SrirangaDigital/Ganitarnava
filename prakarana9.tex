\chapter{9neV parxkaraNa}

\centerline{{\rm\bfseries SUBTRACTION}}
\vskip .3cm

\centerline{{\large\bf vayxvakalanavu.}}
\smallskip

vayxvakalanaveMdare eraDu saMKegaLu hecucx kaDameyAgirutitxralAgi A eraDakUkx iruva vetAyxsavanUnx athavA aMtaravanUnx tiLadukokxLaLxtakakxdudx. idakekx kaLiyuvike eMtalU, vajA bAki eMtalU beVriVju eMtalU vayxvakalanaveMtalU heVLutAtxre. matutx avugaLalilx doDaDx saMKeyanunx shoVdhaniVyaveMtalU saMNa saMKeyanunx shoVdhakaveMtalU kaLadu uLidadadxnunx, uLavu, sheVSa, bAki, aMtaragaLeMdenunxtAtxre.\\

\begin{center}
{\large\bf karxmavu.}
\end{center}

{
\renewcommand{\arraystretch}{1.25}
\tabcolsep=.5cm
\begin{longtable}{|>{$}c<{$}|>{$}c<{$}|>{$}c<{$}|}
\hline
1 - 1 = 0 & 2 - 2 = 0 & 3 - 3 = 0\\
2 - 1 = 1 & 3 - 2 = 1 & 4 - 3 = 1\\
3 - 1 = 2 & 4 - 2 = 2 & 5 - 3 = 2\\
4 - 1 = 3 & 5 - 2 = 3 & 6 - 3 = 3\\
5 - 1 = 4 & 6 - 2 = 4 & 7 - 3 = 4\\
6 - 1 = 5 & 7 - 2 = 5 & 8 - 3 = 5\\
7 - 1 = 6 & 8 - 2 = 6 & 9 - 3 = 6\\
8 - 1 = 7 & 9 - 2 = 7 & 10 - 3 = 7\\
9 - 1 = 8 & 10 - 2 = 8 & 11 - 3 = 8\\
\hline 
4 - 4 = 0 & 5 - 5 = 0 & 6 - 6 = 0\\
5 - 4 = 1 & 6 - 5 = 1 & 7 - 6 = 1\\
6 - 4 = 2 & 7 - 5 = 2 & 8 - 6 = 2\\
7 - 4 = 3 & 8 - 5 = 3 & 9 - 6 = 3\\
8 - 4 = 4 & 9 - 5 = 4 & 10 - 6 = 4\\
9 - 4 = 5 & 10 - 5 = 5 & 11 - 6 = 5\\
10 - 4 = 6 & 11 - 5 = 6 & 12 - 6 = 6\\
11 - 4 = 7 & 12 - 5 = 7 & 13 - 6 = 7\\
12- 4 = 8 & 12 - 5 = 8 & 14 - 6 = 8\\
\hline
7 - 7 = 0 & 8 - 8 = 0 & 9 - 9 = 0\\
8 - 7 = 1 & 9 - 8 = 1 & 10 - 9 = 1\\
9 - 7 = 2 & 10 - 8 = 2 & 11 - 9 = 2\\
10 - 7 = 3 & 11 - 8 = 3 & 12 - 9 = 3\\ 
11 - 7 = 4 & 12 - 8 = 4 & 13 - 9 = 4\\
12 - 7 = 5 & 13 - 8 = 5 & 14 - 9 = 5\\
13 - 7 = 6 & 14 - 8 = 6 & 15 - 9 = 6\\
14 - 7 = 7 & 15 - 8 = 7 & 16 - 9 = 7\\
15 - 7 = 8 & 16 - 8 = 8 & 17 - 9 = 8\\
\hline
10 - 10 = 0 & 11 - 11 = 0 & 12 - 12 = 0\\
11 - 10 = 1 & 12 - 11 = 1 & 13 - 12 = 1\\
12 - 10 = 2 & 13 - 11 = 2 & 14 - 12 = 2\\
13 - 10 = 3 & 14 - 11 = 3 & 15 - 12 = 3\\
14 - 10 = 4 & 15 - 11 = 4 & 16 - 12 = 4\\
15 - 10 = 5 & 16 - 11 = 5 & 17 - 12 = 5\\
16 - 10 = 6 & 17 - 11 = 5 & 18 - 12 = 6\\
17 - 10 = 7 & 18 - 11 = 6 & 19 - 12 = 6\\
18 - 10 = 8 & 19 - 11 = 7 & 20 - 12 = 7\\
\hline
\end{longtable}
}

\medskip
 
\begin{center}
{\large\bf sUtarx.}
\end{center}

\begin{verse}
kaM|| piribele saMKayxsAthxpisi| kiribele saMKayxvanu kiVLebariyuta karxmadi| iruva piribeleya modalino| LirutipApx kiridu beleya modalaMkiyanaM||\\

kaLiduLiduda nAsAthxnadi| keLagaDe gere baradadiDuta| uLidudanelalxva| kaLiyutaladara sAthxnada| keLagaDe bariyutatx poVgu gaNakara matadiM||\\

piribele saMKayxdoLAvadu| kiridAgidaRdarakiVLe irutiha aMkiyu| piridAdare meVlinoLuM| sari badidhxsida shakavanunx kaLiyuta muMduM||\\

kaLiyuva aMkiyoLoMdaM| tiLidoMdanu kUDimeVle peVLida karxmadoLf| kaLiyutapoVgalakxMtara| taLuvadebakukxgaNitadanumatiyiMdaM||

vi|| hecucx beleyuLaLx saMKayxvanunx modalu baradu, adara keLage. kaDame yAda beleyuLaLx saMKayxvanunx AyAya sAthxnagaLa keLage sariyAgi baruva hAge baradu, A meVle piri beleya EkasAthxnada aMkiyalilx kiri beleya EkasAthxnada aMkiyanunx dashaka, shatakAdigaLalilx  AyAya sAthxnada aMkigaLanUnx kaLadu gereV keLage bariyutAtx hoVgabeVku. oMdu veVLe meVlina aMkiyu cikakxdAgiyU  keLagina kaLiya takakx aMkiyu doDaDxdAgiyU idadxre, A meVlina aMkiyalilx $10$ kUDisi koMDu, adaralilx A keLagina aMkiyanunx kaLedu baradu adarAceV sAthxnada aMkiMyanunx kaLiyuvAgeyx adaralilx $1$ sheVrisikoMDu meVlinaMteV kaLiyutAtx hoVgabeVku.
\end{verse}

\begin{center}
{\large\bf hAyxgeMdare.}
\medskip

\begin{tabular}{>{$}c<{$}>{$}c<{$}>{$}c<{$}>{$}c<{$}l}
4 & 5 & 1 & 5 & doDaDx beleyuLaLx saMKeyx\\
3 & 4 & 3 & 4 & cikakx beleyuLaLx saMKeyx\\
\cline{1-4}
1 &  0 & 8 & 1 & sheVSa, athavA aMtaravu\\ 
\cline{1-4}
\end{tabular}
\end{center}

idaralilx meVlina EkasAthxnada aMki $5$ralilx keLagina $4$nunx kaLiyalu=$1$ idanunx gereV keLage baradide. AmeVle meVlina dashasAthxnada aMkiyu $1$  irutetx adaralilx keLagina $3$ hoVguvadilalx. AdadxriMda A $1$ralilx $10$ sheVrisi $11$ deMdu tiLadu keLagina $3$ kaLadu uLida $8$nunx A sAthxnadalilx baradirutatxde. taruvAya adarAce keLagina shatasAthxnada aMkiyAda $4$ralilx $1$ sheVrisi Aguva $5$nunx meVlina $5$ralilx kaLadu uLida sonenxyanunx baradirutatxde. anaMtara meVlina 4 ralilx keLagina 3 kaLadu uLida $1$nunx baradirutatxde.

\begin{center}
{\large\bf tALe.}
\end{center}

\begin{verse}
kaM|| pirisAlananxvadoLagaLi| darituLuvigenavakUDutadaroVLamxtAtx|| kiridananxLaduLida sheVSava| naritaLiyalukxLavu (labadhxnavavaLidasamaM)||\\

vi|| doDaDx sAlanunx $9$riMdA aLadu uLiyuva sheVSadalilx $9$nunx kUDisi adaralilx (cikakx sAlanunx $9$riMdA aLadu vuLiyuva sheVSavanunx kaLadare) uLiyuva aMkiyu (labadhxvanunx $9$riMdA aLadare uLiyuva aMkige) sariyAgira beVku. aMdare leKeKxdalilx tapipxlalxveMbuvadakekx sAkiSx uMTu.\\
hAyxgeMdare $(4+5+1+5)= 15-9=6+9=15$ idaralilx $(3+4+3+4)=14-9=5)$nunx kaLiyalu $=10$ idu labadhxvAda $(1+0+8+1)=10$kekx sariyAgiruvadadxriMda tapipxlalxvu. athavA sheVSavanunx shoVdhakavanUnx kUDisidare shoVdhaniVyada sariyAgirabeVku.\\
\end{verse}

\begin{center}
{\bf\large {\boldmath$4$}neV aBayx udAharaNe}
\end{center}


\begin{longtable}{cccc}
\tabcolsep=4pt
\begin{tabular}{>{$}c<{$}>{$}c<{$}>{$}c<{$}>{$}c<{$}}
\multicolumn{4}{c}{$(1)$}\\[5pt]
1 & 2 & 3 & 4\\
  & 5 & 2 & 3
\end{tabular} & 
\tabcolsep=4pt
\begin{tabular}{>{$}c<{$}>{$}c<{$}>{$}c<{$}>{$}c<{$}}
\multicolumn{4}{c}{$(2)$}\\[5pt]
3 & 4 & 5 & 6\\
2  & 3 & 6 & 8
\end{tabular} &
\tabcolsep=4pt
\begin{tabular}{>{$}c<{$}>{$}c<{$}>{$}c<{$}>{$}c<{$}}
\multicolumn{4}{c}{$(3)$}\\[5pt]
9 & 5 & 4 & 7\\
2  & 4 & 5 & 6
\end{tabular}&
\tabcolsep=4pt
\begin{tabular}{>{$}c<{$}>{$}c<{$}>{$}c<{$}>{$}c<{$}}
\multicolumn{4}{c}{$(4)$}\\[5pt]
9 & 5 & 4 & 5\\
2  & 7 & 5 & 6
\end{tabular}\\[30pt]
\tabcolsep=4pt
\begin{tabular}{>{$}c<{$}>{$}c<{$}>{$}c<{$}>{$}c<{$}}
\multicolumn{4}{c}{$(5)$}\\[5pt]
8 & 4 & 0 & 5\\
2  & 3 & 1 & 4
\end{tabular} &
\tabcolsep=4pt
\begin{tabular}{>{$}c<{$}>{$}c<{$}>{$}c<{$}>{$}c<{$}>{$}c<{$}}
\multicolumn{5}{c}{$(6)$}\\[5pt]
1& 8 & 9 & 2& 5\\
1& 2& 3& 4 & 7
\end{tabular} &
\tabcolsep=4pt
\begin{tabular}{>{$}c<{$}>{$}c<{$}>{$}c<{$}>{$}c<{$}}
\multicolumn{4}{c}{$(7)$}\\[5pt]
1 & 7 & 4 & 3\\
1 & 4 & 5 & 6
\end{tabular} &
\tabcolsep=4pt
\begin{tabular}{>{$}c<{$}>{$}c<{$}>{$}c<{$}>{$}c<{$}}
\multicolumn{4}{c}{$(8)$}\\[5pt]
9 & 5 & 4 & 3\\
7 & 4 & 6 & 5
\end{tabular} \\[30pt]
\tabcolsep=4pt
\begin{tabular}{>{$}c<{$}>{$}c<{$}>{$}c<{$}>{$}c<{$}}
\multicolumn{4}{c}{$(9)$}\\[5pt]
5 & 6 & 3 & 4\\
2 & 5 & 4 & 9
\end{tabular} &
\tabcolsep=4pt
\begin{tabular}{>{$}c<{$}>{$}c<{$}>{$}c<{$}>{$}c<{$}}
\multicolumn{4}{c}{$(10)$}\\[5pt]
2 & 0 & 0 & 0\\
1 & 3 & 4 & 5
\end{tabular} &
\tabcolsep=4pt
\begin{tabular}{>{$}c<{$}>{$}c<{$}>{$}c<{$}>{$}c<{$}}
\multicolumn{4}{c}{$(11)$}\\[5pt]
9 & 0 & 0 & 1\\
5 & 4 & 3 & 4
\end{tabular} &
\tabcolsep=4pt
\begin{tabular}{>{$}c<{$}>{$}c<{$}>{$}c<{$}>{$}c<{$}}
\multicolumn{4}{c}{$(12)$}\\[5pt]
8 & 6 & 4 & 3\\
4 & 5 & 3 & 4
\end{tabular} \\[30pt]
\tabcolsep=4pt
\begin{tabular}{>{$}c<{$}>{$}c<{$}>{$}c<{$}>{$}c<{$}}
\multicolumn{4}{c}{$(13)$}\\[5pt]
7 & 5 & 4 & 6\\
3 & 4 & 9 & 7
\end{tabular} &
\tabcolsep=4pt
\begin{tabular}{>{$}c<{$}>{$}c<{$}>{$}c<{$}>{$}c<{$}}
\multicolumn{4}{c}{$(14)$}\\[5pt]
5 & 0 & 0 & 0\\
4 & 3 & 2 & 5
\end{tabular} &
\tabcolsep=4pt
\begin{tabular}{>{$}c<{$}>{$}c<{$}>{$}c<{$}>{$}c<{$}}
\multicolumn{4}{c}{$(15)$}\\[5pt]
7 & 0 & 0 & 0\\
6 & 3 & 2 & 5
\end{tabular} &
\tabcolsep=4pt
\begin{tabular}{>{$}c<{$}>{$}c<{$}>{$}c<{$}>{$}c<{$}>{$}c<{$}}
\multicolumn{5}{c}{$(16)$}\\[5pt]
9 & 0 & 0 & 0 & 0\\
4 & 3 & 2 & 5 & 4
\end{tabular}\\[30pt]
\tabcolsep=4pt
\begin{tabular}{>{$}c<{$}>{$}c<{$}>{$}c<{$}>{$}c<{$}>{$}c<{$}}
\multicolumn{5}{c}{$(17)$}\\[5pt]
8 & 0 & 0 & 0 & 0\\
6 & 4 & 3 & 5 & 4
\end{tabular} &
\tabcolsep=4pt
\begin{tabular}{>{$}c<{$}>{$}c<{$}>{$}c<{$}>{$}c<{$}>{$}c<{$}}
\multicolumn{5}{c}{$(18)$}\\[5pt]
1 & 9 & 0 & 0 & 0\\
1 & 2 & 5 & 0 & 0
\end{tabular} &
\tabcolsep=4pt
\begin{tabular}{>{$}c<{$}>{$}c<{$}>{$}c<{$}>{$}c<{$}>{$}c<{$}}
\multicolumn{5}{c}{$(19)$}\\[5pt]
1 & 0 & 5 & 0 & 5\\
1 & 0 & 3 & 0 & 6
\end{tabular} &
\tabcolsep=4pt
\begin{tabular}{>{$}c<{$}>{$}c<{$}>{$}c<{$}>{$}c<{$}>{$}c<{$}>{$}c<{$}}
\multicolumn{6}{c}{$(20)$}\\[5pt]
9 & 0 & 8 & 0 & 7 & 0\\
5 & 0 & 0 & 4 & 0 & 3
\end{tabular} \\[30pt]
\tabcolsep=4pt
\begin{tabular}{>{$}c<{$}>{$}c<{$}>{$}c<{$}>{$}c<{$}>{$}c<{$}}
\multicolumn{5}{c}{$(21)$}\\[5pt]
8 & 0 & 0 & 6 & 0\\
5 & 4 & 3 & 0 & 4
\end{tabular} &
\tabcolsep=4pt
\begin{tabular}{>{$}c<{$}>{$}c<{$}>{$}c<{$}>{$}c<{$}>{$}c<{$}>{$}c<{$}}
\multicolumn{6}{c}{$(22)$}\\[5pt]
6 & 0 & 0 & 5 & 0 & 0\\
4 & 0 & 0 & 6 & 0 & 4
\end{tabular} &
\tabcolsep=4pt
\begin{tabular}{>{$}c<{$}>{$}c<{$}>{$}c<{$}>{$}c<{$}>{$}c<{$}>{$}c<{$}}
\multicolumn{6}{c}{$(23)$}\\[5pt]
4 & 5 & 6 & 0 & 4 & 0\\
2 & 3 & 0 & 0 & 6 & 0
\end{tabular} &
\tabcolsep=4pt
\begin{tabular}{>{$}c<{$}>{$}c<{$}>{$}c<{$}>{$}c<{$}>{$}c<{$}>{$}c<{$}}
\multicolumn{6}{c}{$(24)$}\\[5pt]
6 & 0 & 4 & 0 & 0 & 0\\
4 & 0 & 5 & 2 & 3 & 4
\end{tabular} \\[30pt]
\tabcolsep=4pt
\begin{tabular}{>{$}c<{$}>{$}c<{$}>{$}c<{$}>{$}c<{$}>{$}c<{$}>{$}c<{$}}
\multicolumn{6}{c}{$(25)$}\\[5pt]
7 & 0 & 0 & 6 & 0 & 0\\
5 & 3 & 4 & 2 & 5 & 0
\end{tabular} &
\tabcolsep=4pt
\begin{tabular}{>{$}c<{$}>{$}c<{$}>{$}c<{$}>{$}c<{$}>{$}c<{$}>{$}c<{$}}
\multicolumn{6}{c}{$(26)$}\\[5pt]
4 & 5 & 0 & 6 & 0 & 0\\
2 & 3 & 4 & 5 & 6 & 4
\end{tabular} 
\end{longtable}

\begin{align*}
(27)~~ & 545+432-319+105-18.\\
(28)~~ & 432-256+18-6+64-25.\\
(29)~~ & 9257+4639-1254+18-11\\
(30)~~ & 19000-12453-18+128-16.
\end{align*}


\begin{center}
{\bf\large \boldmath{$5$}\,neV aBayx udAharaNe}
\end{center}

\begin{enumerate}[\rm(1)]
\item obabxnu tananxlilxdadx $987650$ rUpAyigaLalilx $36594$ rUpAyigaLanunx baDiDxV sAlakAkxgiyU $1000$ rUpAyigaLanunx dhamARthaRvAgiyU $750$ rUpAyigaLanunx vAyxpArakAkxgiyU koTuTx uLida rUpAyigaLalilx tananx sAlavanunx bage harisi koMDanu. Agalu avanigidadx sAlaveSuTx?
\item obabxnu kirxsatx shakada $1673$neV varuSadalilx maqtavAdanu. Age avana vayasusx $95$ varuSagaLagidadxvu. Adare avanu huTiTxdudx modalogxMDu $1787$ varuSakekx eSuTx vaSaRgaLAgabahudu?
\item oMdu saMKeyxdalilx $890604$nunx sheVrisidare $1908065$ Agutatxde. A saMKeyx yAvadu.
\item obabx $1295432$ rUpAyigaLu vasUlAga takakxdadxkekx oMdu satiR, $15250$ rUpAyigaLU matotxMdAvatiR, $75927$ rUpAyigaLU vasUlAdavu. Adare inUnx eSuTx rUpAyigaLu vasUlAga beVku. heVLu?
\item ibabxru sAhukArurx sheVri oMdu vAyxpArakekx $125000$ rUpAyigaLanunx hAkidaru. adaralilx obabxMdu $95425$ rUpAyigaLadadxre inonxbabxMdeSuTx?
\item oMdu meYlige $1760$ gajagaLu athavA $5280$ aDigaLU athavA $63360$ aMgulagaLU Agiruvavu. Adare adaralilx gajakiMtA ADiyU, aMgulavu eSuTx hecAcxgiruvavu. matutx aMgulagaLigiMtA aDigaLu eSuTx kaDameyAgiruvavu heVLu?
\begin{equation*}
\text{u.}
\begin{cases}
\text{gajakiMtA aDi\;} {\rm 3520\;} \text{hecucx. aMgulavu\;} {\rm 61600\;} \text{hecucx.}\\  
\text{aMgulagaLigiMtA aDi\;} {\rm 58080\;} \text{kaDameyu.}
\end{cases}
\end{equation*}

\item obabx gaqhasathxnu tananxlilxdadx $10.000$ rUpAyigaLalilx maganige $5300$ rUpAyigaLanUnx, maganige $750$\break rUpAyigaLanunx koTaTxnu. A maganu tananx maganige $1927$ rUpAyigaLanunx magaLige $420$ rUpAyigaLanunx koTaTxnu. hAgeV magaLu tananx darxvayxdalilx maganige $500$ rUpAyigaLanunx magaLige $55$ rUpAyiyanunx\break koTaTxLu. Aga A gaqhasathxnalilxyU, avana ibabxru makakxLugaLalilxyU uLida darxvayxgaLeSuTx?

u. gaqhasathxnalilx $3950$ rU. maganalilx $2953$ rU. magaLalilx $200$ rUpAyigaLu.

\item yAva saMKayxdalilx $375$ kekx sheVrisidare $940$ Agutetxde heVLu? \quad u. $565$

\item yAva saMKayxvanunx $12300$ralilx kaLadare $1742$ Agutatxde heVLu? \quad u. $10558.$

\item obabx manuSayxnige $75$ varuSagaLa vayasisxnalilx $36$ varuSada maganU $28$ varuSada magaLU idadxru. avaru taMdeya eSeTxSaTxneV vayasisxnalilx huTiTxdavaru heVLu? matutx matutx A ibabxru makakxLigU e\-SuTx varuSa hecucx kaDame uMTu? \quad u. $39$neV vayasisxnalilx maganU $47$neV vayasisxnalilx magaLU huTiTxdaru. avaribabxrigU $8$ varuSa hecucx kaDame uMTu.

\end{enumerate}


