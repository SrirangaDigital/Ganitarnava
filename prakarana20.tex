\chapter{20neV parxkaraNa}

\begin{center}
{\rm\bfseries COMPOUND DIVISION.}
\vskip .3cm

{\large\bf nAnA vidha belegaLa BAgAkAravu.} 
\vskip .3cm

{\large\bf sUtarx.}
\end{center}

\begin{verse}
kaM|| irutiha BAjayxda baradada| kirutiha BajakavanenxDadi bariyutatxdariM|| dirutiha piribeleyaMkiya| sarisuta baladoLage labadhx bariyutAtxgalf||

uLiduda kirirUpagaLaM| goLisuta kiri beleya kUDi BAgisutadariM|| duLidiha sheVSake modaluM| tiLisida parirUpagoMDu harisuta poVgeY||

vi|| modalu BAjAyxMkigaLanunx baradu Bajakavanunx eDa BAgadoLage baradukoMDu BAjayxda\break doDaDx beleyaMkiyanunx BAgisi laBadhxvanunx bala BAgadoLage baradukoMDu uLida sheVSAMkige kiri beleya rUpavanunx mADikoMDu adaralilx BAjayxdalilxruva kiri beleyanunx sheVrisikoMDu BAgisutAtx punaH sheVSakekx kiri rUpavanunx koDutAtx meVle heVLida parxkArakekx BAgisutAtx\break hoVgabeVku. KaMDagaLanunx mADikoMDAgUyx mADabahudu.
\end{verse}

\begin{center}
{\large\bf hAyxgeMdare.}
\end{center}

udAharaNe--$16$ rU, $14$ ANe $9$ peY idanunx $3$riMda BAgisu.\\
\begin{center}
\begin{tabular}{>{$}c<{$}>{$}c<{$}>{$}c<{$}}
3)\; 16\; \text{rU.} & 14\; \text{ANe}\; 9\; \text{peY} & (5-10-3\\
15 & &\\
\cline{1-1}
1 & &\\
16 & &\\
\cline{1-1}
16 & &\\
14 & &\\
\cline{1-1}
3)\; 30 & &\\
\quad\; 30 & &\\
\cline{1-1}
\;3)\; 009\\
\quad\; 9 &\\
\cline{1-1}
\quad\; 0\; \text{sheVSavu.}
\end{tabular}
\end{center}

idaralilx BAjayxda doDaDx beleyaMki $16$ rUpAyiyanunx $3$ pAlu mADalu BAga $1$kekx $5$ rUpAyi baMtu. sheVSa $1$ rUpAyi uLakoMDitu. adakekx ANeV rUpa koDalu $16$ meVlina Bajakadalilxruva $14$ ANeyanunx sheVrisi koLaLxlu $30$ avanunx BAgisalu BAga $1$kekx $10$ ANe baMtu. sheVSa uLiyalilalx.
