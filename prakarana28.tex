\chapter{28neV parxkaraNa.}

\begin{center}
{\large viSama apUNARMkige pUNARMka athavA BAgAnubaMdha rUpavanunx koDatakakxdudx.}
\vskip .3cm

{\large\bf sUtarx.}
\end{center}

\begin{verse}
kaM|| irutiha viSamA pUNaRke| barutiha BAgAbaMdha rUpaMgoLalA|| pariviDidaMshava CeVdadi| sariBAgisaluLidudaMsha bariCeVdavadeV||

vi|| aMshavanunx CeVdadiMda BAgisu BAga labadx pUNARMkiyAgiyU, sheVSavu aMshavAgiyU,\break CeVdavu modalinadAgiyU tiLadu bari.

\end{verse}

udAharaNe,  $\frac{15}{2}$  I viSama apUNARMkige BAgAnubaMdha rUpavanunx koDu.
$$
\dfrac{15}{2}=7\dfrac{1}{2}
$$ 
idaralilx $15$ aMshavanunx $2$ CeVdadiMda BAgisalu $7$ pUNARMki baMtu. sheVSa uLida $1$ nunx aMshadalUlx modalina CeVda $2$ nunx CeVda sathxLadalUlx baradirutatxde.\\ 
$$
\dfrac{115}{9}= 12 \dfrac{7}{9}\; \text{~ utatxravu.}
$$

\medskip

\begin{center}
{\bf\Large 35neV aBayx udAharaNe.}
\end{center}

\begin{center}
\begin{tabular}{>{$}c<{$}>{$}c<{$}>{$}c<{$}>{$}c<{$}>{$}c<{$}>{$}c<{$}>{$}c<{$}>{$}c<{$}>{$}c<{$}}
(1) & \dfrac{487}{231} & (2) & \dfrac{939}{27} &  (3)  &  \dfrac{1156}{27}  &  (4)  &  \dfrac{5781}{112} \\[20pt]
 (5)  &  \dfrac{5481}{131} & (6) & \dfrac{6489}{194} & (7) & \dfrac{6632}{198} & (8) & \dfrac{6842}{252}\\[20pt]
 (9)  &  \dfrac{6888}{232} & (10) & \dfrac{8949}{344} & (11) & \dfrac{1527}{213} & (12) & \dfrac{1108}{218}\\[20pt]
(13)  &  \dfrac{7422}{165} & (14) & \dfrac{1928}{35} & (15) & \dfrac{14525}{3421} & (16) & \dfrac{10025}{438}\\[20pt]
(17)  &  \dfrac{8648}{37} & (18) & \dfrac{19356}{456} & (19) & \dfrac{26438}{615} & (20) & \dfrac{64521}{315}\\[20pt]
\end{tabular}
\end{center}
