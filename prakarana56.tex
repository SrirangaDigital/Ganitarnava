\vskip 1cm

\begin{center}
{\bf\LARGE 56neV parxkaraNa.}\\
\vskip .3cm
{\rm SIMPLE PROPORTION.}
\vskip .3cm
{\bf tarxyarAshiya gaNitavu.}
\end{center}

terxYrAshi gaNitaveMdare, eraDu beVre beVre jAtigaLa beleyuLaLx saMKayxgaLAgiyU, matotxMdu avugaLalilx yAvadAdarU oMdu jAtiya saMKayxda beleyuLaLxdAdxgiyU idadxre, $1$neV matutx $2$neV saMKayxgaLige Enu saMbaMdha\-virutatxdeyoV aMthA saMbaMdha uLaLx $3$neV saMKayxkekx takakx $4$neV saMKayxvanunx kaMDu hiDiya takakx riVtiyU, idanunx terxYrAshi eMtalU tirx parxmANaveMtalU athavA mUru maneV leKaKxveMtalU anunxtAtxre.

\begin{center}
{\bf upayoVgavu.}
\end{center}

\begin{verse}
kaM|| rAshi tarxya dupayukatxvu| leVsenisuva savxnaRdaMte sAvaRrigihudeY||
rAshitarxyagaLa sUtarxva| leVsAgiye tiLiyuvaMte sulaBadi peVLevx\char'366||

vi|| sherxSaTxvAda BaMgAravu hAyxge elAlx kArayxgaLigU janagaLigU upayoVgisutatxdeyoV hAge, I terxYrAshikada leKaKxda upayoVgavu sakalarigU uMTATxgirutAtxdakAraNa, idara sUtarxvanUnx sulaBavA\-gi heVLutAtxVne.
\end{verse}

\begin{center}
{\bf\large sUtarx.}
\end{center}

\begin{verse}
kaM|| utatxra barutiha jAtiya| netutxta mUraneya sathxLadi bariyuta leKada||
vutatxra hecicxge baruvado| matatxdu kiVLAgi baruva dariyuta mudadiM||

kaM|| piridAdutatxra barutire| piridAduda madayx kirida nAdiyoLiriseY || kiridA dutatxra barutire | kiridAduda madhayx pirida nAdiyoLiriseY||

kaM|| irisuta liVpari padagaLa| sarigava samarAshigoMDu mUreraDanepada|| varitu guNisutatxladanaM| dharisAdiya padadoLAgaladu vutatxraveY||

vi|| utatxra bara beVkAda jAtiya saMKeyxyanunx mUraneV sathxLadalilx baradu koLaLxbeVku. A meVle, A leKaKxdalilx utatxravu hecAcxgi barutatxdeyoV athavA kaDameyAgi barutatxdeyoV eMbuvadanunx AloVcisi, hecAcxgi baruva hAgidadxre, uLida eraDaMkigaLalilx hecAcxdadadxnenxV madhayxdalilx athavA eraDaneV sAthxnavAgi baradu kaDameyAdadadxnunx modalaneV sAthxnadalilx baradukoLaLx beVku.

oMdu veVLe utatxravu kaDameyAgi baruva hAgidadx pakaSxdalilx, kaDameyAda saMKayxvanenxV madhayxdalilx baradu, hecAcxda saMKayxvanunx modalaneV sathxLadalilx baradukoLaLx beVku. A meVle sama rAshigaLanunx mADikoLaLx beVku.
\end{verse}

I riVtiyAgi mADi baradukoMDa meVle avugaLa madhayxgaLalilx parxmANa cinehxgaLanunx baradu anupAtadalilx $4$neV saMKayxvanunx tegiyuvadakekx heVLiruva parxkAra $2$neV $3$neV sAthxnada aMkigaLanunx guNisi, A labadhxvanunx modalaneV sAthxnada aMkiyiMda BAgisa beVku. hiVge mADuvAgeyx PArxkaSx\char'366 guNAkArada riVtiyanunx upayoVgisikoMDu hoDadu mADuvadu sulaBavAgiruvadu. \qq udAharaNegaLu.

\begin{enumerate}[\rm(1)]

\item $8$ maLa vasatxrXkekx $12$ rUpAyi $8$ ANe karxyavAdare, $36$ maLa vasatxrXkekx eSuTx rUpAyi?\\

\begin{tabular}{>{$}c<{$}>{$}c<{$}>{$}c<{$}>{$}c<{$}}
\text{maLa.} & \text{maLa.} & \text{rU.} & \text{ANe.}\\
8 : & 36 :: & 12 & 8\\
&&16 & \\
\cline{3-3}
&&200 & \text{ANegaLu.}
\end{tabular}

Agalu,
 
\qq $\dfrac{36\times {\overset{\displaystyle{25}}{200}}}{\overset{\displaystyle{8}}{1}}= \dfrac{900\; \text{ANe}}{1}\div16=56$ rU. $4$ ANe utatxravu.

\item $5$ maNada $\dfrac{7}{8}$ BAgakekx $4$ rUpAyina $\dfrac{1}{3}$ BAga karxyavAdare, $7$ rUpAyina $\dfrac{5}{6}$ BAgakekx eSuTx maNagaLu heVLu?\\

\begin{tabular}{>{$}c<{$}>{$}c<{$}>{$}c<{$}>{$}c<{$}>{$}c<{$}}
5\times\dfrac{7}{8}=\dfrac{35}{8} & \text{maNa} &\qq\qq \text{rU.} & \text{rU.} & \text{maNa.}\\[10pt]

4 \times\dfrac{1}{3}=\dfrac{4}{3} & \text{rUpAyi} &\qq\qq \dfrac{4}{3} : & \dfrac{35}{6} :: & \dfrac{35}{8}\\[10pt]

7 \times\dfrac{5}{6}=\dfrac{35}{6} & \text{rUpAyi}\\
\end{tabular}

\begin{center}
Agalu,\\

\flushright 
$\dfrac{35}{\overset{\displaystyle{6}}{2}}\times\dfrac{35}{8}\times\dfrac{3}{4}=\dfrac{1125}{64}$\\[3pt]
$=17$ ma. $23\dfrac{1}{8}$ sheVru, utatxravu.
\end{center}

\item $4$ manuSayxru oMdu kelasavanunx $12$ divasagaLalilx mADutAtxre, adeV parxkAra A kelasavu $8$ divasagaLalilx pUreYsa beVkAdare, eSuTx manuSayxrira beVku?

\begin{tabular}{>{$}c<{$}>{$}c<{$}>{$}c<{$}}
\text{di}. & \text{di}. & \text{manuSayxru}\\[3pt]
8 : & 12 :: & 4\\[3pt]
& 6  \\[3pt]
\end{tabular}

\qq\quad $\dfrac{12\times4}{\overset{\displaystyle{2}}{8}}=6 $ \text{ manuSayxru utatxravu.}

\item $4\dfrac{1}{4}$ rUpAyige $15\dfrac{3}{4}$ maNa belalxvAdare, $31\dfrac{1}{2}$ maNa belalxkekx eSuTx rUpAyi karxyavAguvadu?


\begin{minipage}[t]{7cm}
\begin{tabular}{>{$}c<{$}>{$}c<{$}}
\text{idanunx dashamAMshada riVtiya meVle mADu?}\\[10pt]
 4\dfrac{1}{4}=4.25 & \text{rU.}\\[10pt]
 15\dfrac{3}{4}=15.75 & \text{maNa.}\\[10pt]
 31\dfrac{1}{2}=31.5 & \text{maNa}.\\[10pt]
\cline{1-2}
\end{tabular}\\
\begin{tabular}{>{$}c<{$}>{$}c<{$}>{$}c<{$}}
15.75) & 133.875 & (85\; \text{rupAyi}\\
 & \!\!\!126~00 & \qq \text{utatxra.}\\
\cline{2-2}
 & \quad~~ 7875 &\\
 & \quad~~ 7875 &\\
\cline{2-2}
 & \quad~~ 0000&
\end{tabular}
\end{minipage}\quad 
\begin{minipage}[t]{6cm}
\begin{tabular}{>{$}c<{$}>{$}c<{$}>{$}c<{$}>{$}c<{$}}
  \text{maNa}. & \text{maNa}. & \text{rU}.\\
  15.75: & 31.5~ :: & 4.25\\
& \!\!\!\!\!4.25 &\\
\cline{2-2}
& \quad 1575 &\\
& \!\!\quad 630 \\
& \!\!\!260\\
\cline{2-2}
& \quad133.875\\
\end{tabular}
\end{minipage}\\

kelavu leKaKxgaLanunx hiVge dashAMsha riVtiyiMda mADoVNadariMda dashAMsha gaNitada tiLavaLikeyu canAnxgi Agutatxde.
 
\begin{tabular}{>{$}c<{$}>{$}c<{$}>{$}c<{$}>{$}c<{$}}
\text{athavA maNa.} & \text{maNa.} & \text{rU.}\\[10pt]
\qq \dfrac{63}{4} : & \quad \dfrac{63}{2} :: & \quad \dfrac{17}{4} & 
=\dfrac{17}{4}  \times \dfrac{63}{2}=\dfrac{4}{63} \times \dfrac{17}{2}\\[10pt]
&&&  =8\dfrac{1}{2} \text{rU. utatxravu.}
\end{tabular}
\end{enumerate}

\begin{center}
{\bf\large 73neV aBayx udAharaNe.}\\
\vskip .3cm
{\bf sulaBavAda leKaKxgaLU.}
\end{center}

\begin{enumerate}[\rm (1)]
\item $8$ maLakekx $12$ rUpAyi karxyavAdare, $34$ maLakekx eSuTx rUpAyi?

\item $12$ gajakekx $18$ rUpAyi karxyavAdare, $22$ gajakekx?

\item $7$ yADuR banAtige $24\dfrac{1}{2}$ rUpAyi karxyavAdare, $42$ yADiRge eSuTx karxya?

\item $34$ maLakekx $51$ rUpAyi karxyavAdare, $12$ rUpAyige eSuTx maLa?

\item $22$ gajakekx $33$ rUpAyi karxyavAdare, $18$ rUpAyige eSuTx gaja?

\item $42$ yADiRge $147$ rUpAyi karxyavAdare, $24\dfrac{1}{2}$ rUpAyige eSuTx yADuR?

\item $12$ rUpAyige $8$ maLavAdare, $34$ maLakekx eSuTx karxya?

\item $18$ rUpAyige $12$ gajavAdare, $22$ gajakekx eSuTx karxya?

\item $7$ yADiRge $24\dfrac{1}{2}$ rUpAyi karxyavAdare, $147$ rUpAyige eSuTx yADfR?

\item $5$ maNakekx $12\dfrac{1}{2}$ rUpAyi karxyavAdare, $50$ rUpAyige eSuTx maNa?

\item $20$ maNakekx $50$ rUpAyi karxyavAdare, $12\dfrac{1}{2}$ rUpAyige eSuTx maNa?

\item $1$ tiMgaLige $18\dfrac{3}{4}$ rUpAyi saMbaLavAdare, $7$ divasakekx eSuTx?

\item $7$ divasakekx $4$ rUpAyi $6$ ANe kUliyAdare, $1$ tiMgaLige eSATxyitu?

\item $13$ divasakekx $29\dfrac{1}{4}$ rUpAyi saMpAdisidare, $5$ divasada saMpAdane eSuTx?

\item $5$ divasakekx $11\dfrac{1}{4}$ ANe kUli sikikxdare, $13$ divasakekx eSuTx kUli?

\item $7$ divasakekx $35$ rUpAyi $14$ ANe saMbaLavAdare, $28$ divasakekx eSuTx rU. saMbaLa?

\item $28$ divasada tiMgaLige $143$ rUpAyi $8$ ANeyAdare, $13$ divasakekx?

\item $4$ maNakekx $67\dfrac{1}{2}$ rUpAyi karxyavAdare, $30$ sheVrige eSuTx karxya?

\item $3$ dhaDiyakekx $16$ rUpAyi $14$ ANe karxyavAdare, $4$ maNakekx eSuTx karxya?

\item $1$ KaMDagakekx $11$ rUpAyi $14$ ANe karxyavAdare, $3$ koLagakekx eSuTx karxya?

\item $3$ koLagakekx $1$ rUpAyi $12$ ANe $6$ kAsu karxyavAdare, $16$ koLagakekx eSuTx karxyavu?

\item $5$ varahA tUka BaMgArakekx $22$ rUpAyi $10$ ANe $6$ kAsu karxyavAdare, $1$ varahA tUkakekx eSuTx karxya?

\item $3$ varahA tUkakekx $13$ rUpAyi $9$ ANe $6$ kAsu karxyavAdare, $4\dfrac{1}{2}$ rUpAyi tUkakekx eSuTx karxya?


\item $5$ pwMDf BAravuLaLx oMdu beLiLxV pAterxya karxyavu $7$ pwMDf $5$ SililxMgf $10$ pe\char'366sf karxyavAdare, $8$ pwMDf $5$ au\char'366sf $15$ peninxveVTf BAravuLaLx pAterxya karxya eSATxgutatxde?

\item $14$ pwMDige $30$ rUpAyi karxyavAdare, $56$ pwMDige eSuTx karxya?

\item $72$ maLakekx $64$ rUpAyi karxyavAdare, $90$ yADiRge eSuTx karxya?

\item $16$ rUpAyige $18$ maNagaLAdare, $480$ rUpAyige eSuTx maNa?

\item $34$ gAla\char'366\  dArxkASx rasakekx $510$ rUpAyi karxyavAdare, $170$ rUpAyige eSuTx gAla\char'366?

\item $36$ ekarege $540$ rUpAyi karxyavAdare, $25$ ekare $12$ guMTege eSuTx?

\item $15$ janakekx $1$ kelasakekx $105$ divasa beVkAdare, $95$ janagaLige adeV kelasakekx eSuTx divasa beVku?
\end{enumerate}

\begin{center}
{\bf\large 74neV aBayx udAharaNe.}\\
\vskip .3cm
{\bf savxlapx sulaBavAda leKaKxgaLU.}
\end{center}

\begin{enumerate}[\rm (1)]
\item $3$ tAsugaLalilx $4$ haradAri naDadare, $12$ tAsugaLalilx eSuTx naDadAnu?

\item $100$ rUpAyige $4\dfrac{1}{2}$ rUpAyi baDiDxyAdare, $428$ rUpAyi $310$ ANege eSuTx baDiDx?

\item $27\dfrac{1}{2}$ yADiRge $19$ rUpAyi $14$ ANe $6$ peY karxyavAdare, $196$ yADiRge eSuTx karxya?

\item $18$ janaru $28$ divasagaLalilx oMdu kelasavanunx mADidare, adeV kelasavanunx $14$ divasadalilx pUreYsa beVkAdare, eSuTx janagaLira beVku?

\item $1$ divasakekx $9$ haradAri meVrige naDiyutAtx tananx sathxLakekx $60$ divasagaLige talapidare, adeV parxyANavanunx parxti divasavU $12$ meYlf parxkArakekx naDadukoMDu hoVdare, eSuTx divasa beVku?

\item obabxna $5$ hejejxgaLalilx $12$ yADfR $1$ PUTf $6$ iMcf BUmiyAdare, avana $1420$ hejejxgaLalilx eSuTx yADuR BUmiyAguvadu?

\item $9$ tiMgaLige $275$ rUpAyina baDiDx shikikxdare, $3850$ rUpAyina baDiDx eSuTx divasakekx shikukxvadu?

\item obobxbabxnige parxti divasavu $2$ pAvina parxkArakekx koTaTxre, $5$ tiMgaLigAguvaSuTx kALugaLa saMgarxha virutatxde, adeV dhAnayxvu $8$ tiMgaLa varigU poreYsuva hAge parxti divasavU obobxbabxnige eSuTx pAvina parxkArakekx koDa beVku?

\item $50$ rUpAyige $4$ KaMDi goVdhiya dhAraNeyiruvAgeyx $1$ roTiTxge $4$ ANe $4$ peY karxya bidadxre, goVdhiya karxyadalilx $20$ rUpAyi kaDimeyAdare, $1$ roTiTxge eSuTx karxya biVLuvadu?,

\item $10$ rUpAyi maja\char'371\char'301riV koTaTxre $7$ maNa BAravanunx $63$ meYlf tegadukoMDu hoVgutAtxne. Adare, adeV maja\char'371\char'301rige $21$ maNa BAravanunx eSuTx meYlf hotutxkoMDu hoVga beVku?

\item $\dfrac{5}{7}$ gaja kelasavanunx $\dfrac{5}{6}$ divasadalilx mADidare, $\dfrac{1}{8}$ divasadalilx eSuTx kelasavanunx mADAyxru?

\item $1$ moLa $\dfrac{3}{8}$ra karxyavu $1$ rUpAyina $\dfrac{2}{5}$ Adare, $1$ moLada $\dfrac{5}{16}$ra karxyaveSuTx?

\item $\dfrac{3}{9}$ GaLigeyx, $13$ meYlfna $\dfrac{5}{6}$ BAgavanunx naDadare, $9$ divasakekx eSuTx meYlfgaLanunx naDidAnu? 

\item $3\dfrac{1}{2}$ra $\dfrac{5}{6}$ maNa tupapxkekx $5\dfrac{3}{4}$ra $\dfrac{2}{3}$ rUpAyi karxyavAdare, $16$ rUpAyina $\dfrac{4}{5}$ ra $\dfrac{3}{4}$kekx eSuTx maNavu?

\item $700$ meYlf parxyANavanunx $20$ divasada $\dfrac{2}{3}$ralilx hoVgi baMdare, $15$ rU. $15$ ANeya $\dfrac{2}{3}$ inAM koDa takakx kulxpatxvidadxre, avanu $25$ divasada $\dfrac{2}{5}$ralilx baMdare, eSuTx rU. koDa beVku?

\item oMdu haDagada $\dfrac{3}{16}$ BAgakekx $12000$ rUpAyi karxyavAdare, adeV haDagada $\dfrac{2}{5}$ BAgakekx eSuTx karxya?

\item $348 \dfrac{3}{5}$ cadara gaja kelasakekx $1250$ rUpAyi KacARgidadxre, dara gajakekx eSuTx KacuR?

\item $10\dfrac{1}{8}$ maNakekx $25\dfrac{1}{2}$ rUpAyi karxyavAdare, $1$ KaMDi $2\dfrac{3}{4}$ maNa $5\dfrac{1}{3}$ sheVrige eSuTx?

\item $100$ rUpAyige $1$ tiMgaLige $8\dfrac{1}{2}$ ANe baDidxyAdare, $525$ rUpAyi $8$ ANege $9\dfrac{3}{4}$ tiMgaLige eSuTx baDiDx?

\item $1$ KaMDi $3$ maNa $5$ sheVru kaDelxge $27$ rupAyi $10$ ANe $9$ peY karxyavAdare, $221$ rUpAyi $6$ ANege eSuTx kaDelx shikukxvadu?

\item $120$ gaja agalavuLaLx BUmiyu bahaLa udadxkikxrutatxde. adaralilx $1$ cadara guMTeTx aMdare, $121$ gaja agala $121$ gaja udadxdaSuTx tegadukoMLaLx beVkAdare, eSuTx gaja udadxvanunx tegadukoLaLx beVku?


\item $12$ PUTf udadx $15$ PUTf agalavAda $1$ koThaDige hAsuvadakekx jaMKAne beVkAgirutatxde. Adare, A jaMKAneyu $3$ PUTf $4$ iMcf agalavAgi bahaLa udadxkekx shikukxtatxde. Adare adaralilx eSuTx udadxvAgi tegadukoMDare A koThaDige sariyAguvadu?
\end{enumerate}

\begin{center}
{\bf\large 75neV aBayx udAharaNe.}\\
\vskip .3cm
{\bf savapx kaSaTxvAda parxshenxgaLU.}
\end{center}

\begin{enumerate}[\rm(1)]
\item kelavu janagaLu sheVri $1$ kelasavanunx $75$ divasagaLalilx mADutAtxre. Adare, inUnx $15$ janagaLu hecAcxgi sheVri $50$ divasagaLalilx mADidaru. Adare modalina janagaLeSuTx?

manuSayx.

$
\left.
\begin{tabular}{>{$}c<{$}>{$}c<{$}>{$}c<{$}}
0 & 75\; \text{di.}\\
15 & 50\; \text {di.}\\
\cline{2-2}
& 25\; \text {di.}
\end{tabular}
\right \{
\begin{tabular}{>{$}c<{$}>{$}c<{$}>{$}c<{$}>{$}c<{$}}
\text{divasakekx.} & \text{di.} &  \text{manu}\\
25 & 75 & 15\\
&&&15\times3=45 \text{~ manuSayxru utatxravu.}
\end{tabular}
$

idaralilx $15$ janagaLu hecAcxdadadxriMda $25$ divasa muMcitavAgi A kelasa mugadirutetx. AdadxdariMda $25$ divasakekx $15$ manusayxru parxmANa shikikxda hAgAyitu. adara AdhAradiMda melakxMDa parxkArakekx modalina manuSayxranunx kaMDu hiDiya beVku.

\item $1$ kelasavanunx (a) aMbuvanu $8$ divasadalilx mADutAtxne. adeV kelasavanunx (bi) eMbuvanu $15$ divasadalilx mADutAtxne. Adare, avaribabxrU sheVri mADidare, eSuTx divasagaLalilx mADuvaru.

\begin{tabular}{>{$}c<{$}>{$}c<{$}>{$}c<{$}>{$}c<{$}}
\text{divasakekx.} & \text{kelasavAdare.} & \text{divasakekx}\\[10pt]
8 : & 1 :: & 1=\dfrac{1}{8} & \text{kelasavanunx (a) eMbuvadanu $1$ divasakekx mADutAtxne.}\\[10pt]
15 : & 1 :: & 1=\dfrac{1}{15} & \text{kelasavanunx (bi) eMbuvadanu $1$ divasakekx mADutAtxne.}\\
\end{tabular}\\[10pt]

Agalu,\quad  $\dfrac{1}{8}+\dfrac{1}{15}=\dfrac{15+8}{120}=\dfrac{23}{120}$\; kelasavanunx avaribabxrU sheVri $1$ divasakekx mADutAtxre.\\[10pt]

\qq Adare pUrA kelasakekx $\dfrac{120}{23}=5\dfrac{5}{23}$\; divasa utatxravu.

ideV parxkArakekx $3, 4, 5, 6$ janagaLigAdarU mADa bahudu. matutx oMdu kAraMjige eSATxdarU bacacxlugaLidadxvu, avugaLu eSeTxSuTx hotitxgAdarU tuMbutitxdadxvu. A elAlx bacacxlagaLanUnx Eka kAladalilx biTaTxre, eSoTxtitxge tuMbiVteMba leKaKxgaLu sahA Agutatxve.

\item {\rm A B} eMbuvaribabxrU $6$ divasagaLalilx $8$ ANegaLanUnx {\rm A C} eMbuvaru $4$ divasagaLalilx $12$ duDuDxgaLanUnx {\rm B C} eMbuvaru $5$ divasagaLalilx $8$ ANe $4$ kAsugaLanunx saMpAdisutAtxre. Adare, divasakekx obobxbabxra parxteyxVka saMpAdanegaLeSuTx?

\begin{center}
\qq\; di.  ANe, \quad di. \quad ANe.\\
\begin{tabular}{>{$}l<{$}>{$}l<{$}}
{\rm A + B} = 6 : 8 & :: 1 = 1 -4\\
{\rm A + C} = 4 : 4 & :: 1 = 1 -0\\
\end{tabular}\\[3pt]
\quad\; A. \quad kA\\
\begin{tabular}{>{$}l<{$}>{$}l<{$}}
{\rm B + C} = 5 : 8 & 4 :: 1 = 1-8
\end{tabular}
\end{center}

$
\left.
\begin{tabular}{>{$}c<{$}>{$}c<{$}}
\text {Agalu}, &   \qq$--------------------$\\
& \qq\qq \text{A. kA.}\\
& \qq{\rm A + B} = 1-4\\
&  \qq{\rm A + C} = 1-0\\
&  \quad$-----------------------$\\
& 2{\rm A+B+C}=2-4\\
&\qq {\rm B+C}=1-8\\
\text{hoVgalAgi} & \qq $--------------------$\\
\quad \text{sheVSavu} & 2{\rm A} \quad =0-8\\
& \quad\; {\rm A} \quad =4\text{~~~ peY.}\\
& $---------------------$
\end{tabular}
\right\}
\begin{tabular}{>{$}c<{$}>{$}c<{$}}
\text {matutx}, &  $--------------------$\\
& \qq\;\text{A. peY.}\\
& {\rm A + C} = 1-0\\
& {\rm A\quad\; } = 0-4\\
\text{bAki} & $--------------------$\\
\text{matutx}& \dfrac{{\rm C \qq=8\text{peY.}}}{\qq\qq \text{A. peY.}}\\
& {\overset{\displaystyle{{\rm B + C}= 1-8}}{\quad\;{\rm C}= \quad\;8}}\\
\text{bAki} &$--------\;------$\\
& {\rm B}=\quad\; 1\text{ANe.}\\
\end{tabular}
$

Aga utatxravu,

$
\left.
\begin{tabular}{>{$}c<{$}>{$}c<{$}>{$}c<{$}>{$}c<{$}>{$}c<{$}>{$}c<{$}}
{\rm A} & \text{eMbuvana\; saMpAdaneyu} & 1 & \text{divasakekx} & 4 &  \text{peY.}\\
{\rm B}& \text{eMbuvana \;saMpAdaneya} & 1 &     '' &   1 &  \text{ANe.}\\
{\rm C} &  \text{eMbuvana\; saMpAdaneyu} &  1  &   '' &   8 & \text{peY.}
\end{tabular}
\right \} \text{utatxravu.}
$

iMthA leKaKxgaLelAlx ideV riVtiyAgi mADabeVku.

\item (a) eMbuvanu oMdu kelasada $\dfrac{5}{9}$ BAgavanunx $10$ divasagaLalilx mADi, A meVle (ba) eMbuvananunx sahAyakekx karadukoMDu elAlx kelasavanUnx $3$ divasadalilx pUreYsidanu. Adare, A kelasavanunx avaribabxru sheVri, athavA obobxbabxreV mADuvadakekx eSuTx divasagaLu beVku?

\qq \begin{tabular}{>{$}c<{$}>{$}c<{$}>{$}c<{$}>{$}l<{$}}
&& \text{di.} & \text{ke.} \\[5pt]
&\dfrac{5}{9}\; \text{kelasakekx}\; : &\quad \dfrac{10}{1}\; ::& 1=\dfrac{90}{5}=  18 \text{ divasa (a) eMbuvanu}\\
 &&& \qq\quad\quad1\; \text{kelasavanunx obabxneV maDuva}\\ 
&&& \qq\quad\quad\text{dakekx Aguva divasagaLU.}
\end{tabular}

\begin{tabular}{>{$}c<{$}>{$}c<{$}>{$}l<{$}}
\text{uLida kelasa}\; \dfrac{4}{9}\; \text{a, ba ivaribabxrU} & \text{di}.& \text{ke}.\\
\text{mADuvadakekx} & : 3 :: & 1=\dfrac{27}{4}=6\dfrac{3}{4}\quad  \text{divasa (a, ba) }\\ 
&&\qq\quad \text{ivaribabxrU sheVri $1$\;kelasa}\\
&& \qq\quad \text{vanunx mADuvadakekx}\\ 
&&\qq\quad \text{beVkAda divasagaLU}.
\end{tabular}

\begin{tabular}{>{$}c<{$}>{$}c<{$}>{$}c<{$}>{$}c<{$}}
\text{Agalu}, & 18\; \text{divasakekx (a) eMbuvanu mADutAtxne.}\\
& 6\tfrac{3}{4}\; \text{divasakekx (a, ba) eMbuvaru sheVri mA}\\
& \qq\text{DutAtxraSeTx. idariMda avara parxteyxV}\\
& \qq\text{kavAgi mADa takakx kelasavanunx kaMDu}\\
&\text{hiDiya beVku.}
\end{tabular}

\begin{tabular}{>{$}c<{$}>{$}c<{$}>{$}c<{$}>{$}l<{$}>{$}c<{$}}
\text{hAyxgeMdare,} & \text{di.} & \text{ke.} & \text{di.}\\
& 18 &  :\quad 1 \quad:: & 1=\dfrac{1}{18} & \text{kelasa a eMbuvanu mADa takakxdudx.}\\[15pt] 

& 6\dfrac{3}{4} &:\quad  1 \quad:: & 1=\dfrac{4}{27} & \text{kelasa a, ba eMbuvaribabxrU sheVri mADa takakxdudx.} 
\end{tabular}\\[15pt]

\begin{tabular}{>{$}c<{$}>{$}c<{$}>{$}c<{$}}
\text{Agalu}, & \dfrac{4}{27}-\dfrac{1}{18}=\dfrac{8-3}{54}=\dfrac{5}{54}&\text{kelasavu ba eMbuvanobabxneV}\, 1\, \text{divasadalilx mADa takakxdAdxyitu.}\\[15pt]
& & \text{Adare, pUrA kelasakekx}\quad \dfrac{54}{5}=10\dfrac{4}{5}\quad \text{divasavu.}\\[15pt]
\end{tabular}

\begin{tabular}{>{$}c<{$}@{\;}>{$}l<{$}}
& \quad\text{utatxravu.} \\[5pt]
\text{a}& \text{eMbuvanige\; $18$\; divasa.}\\[2pt]
\text{ba} & \text{eMbuvanige\; $10\tfrac{4}{5}$\; divasa.}\\[2pt]
\text{a},  & \text{ba eMbuvarige\; $6\tfrac{3}{4}$ divasa.}\\[2pt]
\end{tabular}
\item a, ba eMbuvaribabxrU oMdu kAvalanunx gutitxgeyx tegadukoMDaru. adu $6$ kuduregaLige $30$ divasada meVvAgiyU, $8$ hasugaLige $72$ divasagaLa meVvAgiyU itutx. Aga avaribabxrU sheVri $1$ kudureyanUnx $1$ hasuvanunx biTuTxre, eSuTx divasagaLa meVvAga bahudu?

$6$ kuduregaLige : $30$ divasa :: $1$ kudurege = $180$ divasa.

$8$ hasuvige \quad: $72$ divasa :: $1$ hasuvige = $576$ divasa.

Agalu, $1$ kudaregU $1$ hasuvigU eSeTxMdu noVDa beVku. hAyxgeMdare,

\begin{tabular}{>{$}c<{$}>{$}c<{$}>{$}l<{$}>{$}l<{$}}
\text{di.}  & \text{kAvalu.} & \text{di.} \\[10pt]
180 &:\quad  1 \quad:: & 1 = \tfrac{1}{180} & 1\; \text{kudure}\; 1\; \text{divasakekx meVyi takakxdudx.}\\[10pt]

576 & :\quad  1 \quad:: & 1 = \tfrac{1}{576} & 1\; \text{hasu}\; 1\; \text{divasakekx meVyi takakx BAgavu.}
\end{tabular}\\[10pt]

\begin{tabular}{>{$}c<{$}>{$}c<{$}>{$}l<{$}}
\text{Iga} & \dfrac{1}{180}+\dfrac{1}{576}=\dfrac{16+5}{2880}=\dfrac{21}{2880}& 1\; \text{divasakekx}\; 1\; \text{kudure}\\
&& \text{yU\; $1$\; hasuvU saha}\\ 
&&\text{meVyi takakx BAgavu.}\\[5pt]
& \text{Aga pUrA meVyuvadakekx}\; \tfrac{2880}{21}= & 137\tfrac{1}{7}\; \text{divasa utatxravu.}
\end{tabular}

\item $12$ manuSayxru $30$ divasagaLalilx $120$ gaja BUmiyanunx AgiyutAtxre. matutx $18$ manuSayxru $3$ divasadalilx $54$ gaja kelasavanunx mADutAtxre. Adare, avarelAlx sheVri $462$ gaja kelasavanunx eSuTx divasagaLalilx mugishAyxru, heVLu?

\begin{tabular}{>{$}c<{$}>{$}c<{$}>{$}l<{$}>{$}l<{$}>{$}l<{$}}
\text{Di.} & \text{gaja} & \text{di.}\\
30 & :\quad 120 \quad:: & 1= & \;4\; \text{gaja} & 12\; \text{manuSayxru mADa takakxdudx.}\\ 
\;\,3& :\quad  \;\;54 \quad:: & 1= & 18\; \text{gaja} & 18\; \text{manuSayxru mADa takakxdudx.}\\[-7pt]
& & & $--------$ & $-----$ \\[-7pt]
& & & 22\; \text{gaja} &30\; \text{manuSayxru mADa takakxdAdxyitu.} 
\end{tabular}

\begin{tabular}{>{$}c<{$}>{$}c<{$}>{$}c<{$}>{$}c<{$}>{$}l<{$}}
\text{Agalu}, & \text{gajakekx} & \text{di}. & \text{gajakekx}\\
& 22 : & 1 :: & 462 & =21\; \text{divasa utatxravu.}
\end{tabular}

\item obabxnu tananx $6$ divasada pArxpitxyanunx $8$ divasadalilx KacuR mADutAtx baralAgi, $128$ divasagaLalilx $1440$ rUpAyigaLu shilukx uLadavu. Agalu avana pArxpitx, matutx KacuR eSuTx?

idaralilx $6$ divasakekx $1$ rUpAyi pArxpitx yeMtalU $8$ divasakekx $1$ rUpAyi KacuR yeMtalU iTuTx koLoLxVNa.

\quad \begin{tabular}{>{$}c<{$}>{$}c<{$}>{$}c<{$}>{$}l<{$}>{$}l<{$}}
\text{Agalu}, & \text{di} & \text{pArxpitx}. & \text{di.}\\[8pt]
& 6 & :\quad  1 \quad:: & 128 & =\tfrac{128}{6}\; \text{pArxpitxyAyitu.}\\[10pt]
& 8 & :\quad  1 \quad:: & 128 & =\tfrac{128}{8}\; \text{KacARyitu.}
\end{tabular}\\[10pt]

\begin{tabular}{>{$}l<{$}>{$}l<{$}>{$}l<{$}>{$}l<{$}>{$}l<{$}>{$}l<{$}}
\text{pArxpitxyalilx KacaRnunx} & 128 & 128 & 512-384 & 128\\[-7pt] 
&  $-----$ \;- & $-----$ \;= & $--------------$ \;= & $-----$ & \text{sheVSavu uLiVtu}.\\[-7pt]
 \qq\quad\; \text{kaLadare} & \quad 6 & \quad 8 & \quad\;\; 24 &  \;24
\end{tabular}\\[7pt]

\begin{tabular}{>{$}c<{$}>{$}c<{$}>{$}l<{$}}
\text{saMgarxhakekx} & \text{pArxpitx} & \text{rUpAyige}\\[10pt]
\tfrac{128}{4} & :\quad \tfrac{1}{1} \quad:: & \tfrac{1440}{1}  = 3\times1\times90=270\; \text{rUpAyigaLu}\\
&&6 \text{ divasada pArxpitxyu matutx} 8 \text{ divasada KacuR, utatxravu.}\\
\end{tabular}

\item $10$ manuSayxru $12$ gaja kelasavanunx $16$ avarinalilx mADutAtxre. $12$ manuSayxru $9$ gaja kelasavanunx $5$ avarinalilx mADutAtxre. Adare, $100$ gaja kelasavanunx avaribabxrU sheVri eSuTx avarinalilx mADutAtxre, heVLu?

\begin{tabular}{>{$}c<{$}>{$}c<{$}>{$}c<{$}>{$}c<{$}}
\text{avarfge} & \text{gaja} & \text{avarfge} \\[10pt]
16 &:\quad 12 \quad:: & 1=\tfrac{3}{4} & \text{gaja kelasa}\; 10\; \text{manuSayxru mADa takakxdudx}.\\[10pt]
15 & :\quad  \;\;9 \quad:: & 1=\tfrac{3}{5} & \text{gaja kelasa}\; 12\; \text{manuSayxru mADa takakxdudx}.\\[10pt]
\end{tabular}

\;\begin{tabular}{>{$}c<{$}>{$}c<{$}}
\dfrac{3}{4}+\dfrac{3}{5}=\dfrac{15+12}{20}=\dfrac{27}{20} & \text{gaja kelasa elAlx manuSayxru sheVri}\\ 
& 1\; \text{avarfnalilx mADa takakx kelasavu.}
\end{tabular}\\[10pt]

\quad \begin{tabular}{>{$}c<{$}>{$}c<{$}>{$}c<{$}>{$}c<{$}}
\text{ke.} & \text{avarf} & \text{gaja kelasakekx} \\[10pt]
\dfrac{27}{20} &:\quad \dfrac{1}{1} \quad:: & \dfrac{100}{1}=74\dfrac{2}{27} & \text{avarf utatxravu.}\\
\end{tabular}\\[10pt]

\item obabxnu tananx holavanunx kuyuvadakekx $1$ tiMgaLige $60$ manuSayxranunx gotutx mADidadxnu. avaru $10$ divasa kelasavanunx mADida meVle inUnx $40$ manuSayxranunx hecAcxgi iTuTxkoMDanu. Aga avanu tananx holada kelasavanunx eSuTx divasakekx pUreYsira bahudu?

\begin{tabular}{>{$}c<{$}>{$}c<{$}>{$}c<{$}>{$}c<{$}}
\text{di.} & \text{kelasa} & \text{di.} \\[5pt]
30 : & 1 :: & 10 & = \tfrac{1}{3}\; \text{kelasavanunx mADidadxru}\\[10pt]

\text{kelasakekx} & \text{divasa} & \text{bAki iruva} \\[5pt]
\tfrac{1}{3} & :\quad   \tfrac{10}{1} \quad:: & \quad \tfrac{2}{3}\; \text{kelasakekx} & =20\; \text{divasa idu } 60\; \text{manuSayxru} \\
& & & \text{mADa takakx divasavu.}\\[10pt]

\text{manuSayxrige} & \text{divasa} & \text{manuSayxrige.} \\[5pt]
60 &:\quad  20 \quad:: & \quad 100  & = 
12\; \text{divasa utatxravu.}\\[10pt]
\end{tabular}

\item obabxnu $16$ divasakekx eSoTxV rUpAyigaLanunx saMpAdisutAtxne. matutx $12$ divasakekx $8$ rUpAyigaLanunx KacuR mADutAtxne. ideV riVtiyAgi $2$ vaSaR naDisi noVDalAgi, $2000$ rUpAyi shilukx uLaditutx. Adare avanu $16$ divasakekx 
eSuTx rUpAyigaLanunx saMpAdisutitxra bahudu heVLu?

idaralilx $2$ vaSaRda = $720$ divasagaLU.

\begin{tabular}{>{$}l<{$}>{$}c<{$}>{$}l<{$}}
\text{di.} & \text{rU. KacuR} & \text{di.}\\[5pt]
12  \quad:&  8\qq  :: &720=480 \text{ rUpAyi KacuR}
\end{tabular}

$
\left.
\begin{tabular}{>{$}c<{$}>{$}c<{$}>{$}c<{$}>{$}c<{$}}
\text{di.} & \text{rU. pArxpitx} & \text{di.} & \text{rU. utatxra}\\[5pt]
720 \quad: & 2480 \quad:: & 16  & =55\tfrac{1}{9}\\
\end{tabular}
\right\}
\begin{tabular}{>{$}l<{$}}
2000\; +\; 480=\\
2480\; \text{rUpAyi}\\
2\; \text{vaSaRda pArxpitxyAyitu.}
\end{tabular}
$

\item obabx vAyxpAriyu $3$ rUpAyige $4$ maNadaMte tegadukoMDu $4$ rUpAyige eSoTxV maNada hAge mAridanu. I meVrige $900$ rUpAyigaLa vAyxpAravanunx mADalAgi $700$ rUpAyigaLu lABa shikikxdavu. Agalu avanu $4$ rUpAyige eSuTx maNagaLa hAge mArira bahudu.

\begin{tabular}{>{$}c<{$}>{$}c<{$}>{$}c<{$}>{$}l<{$}}
\text{rUpAyige} & \text{maNa} & \text{rUpAyige} & \text{maNagaLu tegadu koMDadudx.} \\[5pt]
3 &:\quad  4 \quad:: & 900 & =1200\\
\end{tabular}\\

Agalu $900\; +$ naPe $700=1600$ rUpAyigaLige, A tegadukoMDaMthA $1200$ maNagaLanunx mArida hAgAyitu. AdadxriMda,

\begin{tabular}{>{$}c<{$}>{$}c<{$}>{$}c<{$}>{$}c<{$}>{$}c<{$}}
\text{rUpAyige maNa.}  && \text{rU.} & \text{maNa.}\\[5pt]
1600 : 1200 &:: & 4 = & 3 & \text{utatxravu}.\\
\end{tabular}\\[5pt]

\item $1$ etatxnUnx matUtx $5$ rUpAyigaLanUnx koTaTxre, $1$ kudare karxyavAgutatxde. matutx $8$ etutxgaLanUnx $4$ rUpAyigaLanUnx koTaTxre, $6$ kudare karxyavAgutatxde. Adare $1$ etatxnUnx $1$ kudareyanUnx koTaTxre, eSuTx rUpAyigaLu shikakx bahudu?

\begin{tabular}{>{$}c<{$}>{$}c<{$}>{$}c<{$}>{$}c<{$}>{$}c<{$}>{$}c<{$}>{$}c<{$}>{$}c<{$}>{$}l<{$}}
\text{kudarege} && \text{etutx.} & \text{rU.} && \text{kudarege.} && \text{etutx} & \text{rUpAyigaLAdavu.}\\[5pt]
1 & : & 1 & 5 & :: & 6 & = & 6 & 30\\[5pt]
& & &\text{matutx} &  6 & \text{kudare}& = & 8 &  \;4\;  \text{iruvadadxriMda}\\ 
&&&&&&&& \text{uMTAguva vetAyxsavu.}
\end{tabular}

\begin{center}
\begin{tabular}{>{$}c<{$}>{$}c<{$}>{$}c<{$}>{$}c<{$}>{$}l<{$}>{$}c<{$}>{$}c<{$}}& 2\; \text{etutx}&&26& \text{rUpAyi.}\\
& 1\; \text{etitxna}&=&13 & \text{rUpAyi.}\\
\cline{2-4}
\text{matutx} & 13+5 & =& 18 & \text{rUpAyi}\; 1\; \text{kudare karxyavu.}\\ 
&& \multicolumn{4}{c}{\text{oTuTx} $31$ \text{rUpAyi utatxravu.}}
\end{tabular}
\end{center}

\item nAyi matutx mola ivugaLa madhayxdalilx $50$ yADfR aMtara virutatxde. Adare nAyiyu $1$ minUyxTfnalilx $8$ yADfRnUnx molavu $1$ minUTfnalilx $6$ yADfRnUnx naDiyatakakxvugaLAgiyU, oMdanonxMdu hiDiyuvadakAkxgiyU horaTare tiVvarxgamanavuLaLx nAyige molavu eSuTx hotitxge shikukxvadu? matutx eSuTx yADfR dUradalilx shikukxvadu?

\begin{tabular}{>{$}c<{$}>{$}c<{$}>{$}c<{$}>{$}l<{$}}
1\; \text{minUyxTige} & = & 8  & \text{yADfR nAyi naDiyutatxde.}\\
1\; \text{minUyxTige} & & 6  & \text{yADfR molavu naDiyutatxde.}\\
\cline{1-3}
1\; \text{minUyxTige} & & 2  & \text{yADfR vetAyxsa uMTAda hAgAyitu.}
\end{tabular}

\qq\begin{tabular}{>{$}c<{$}>{$}c<{$}>{$}c<{$}}
\text{Agalu,} & \text{yADfR vetAyxsakekx.} & \text{yADfR nAyi naDiyutatxde.}\\[3pt]
& \quad 2 \qq\quad: & \quad8 \qq\quad::\\[3pt]
& \text{yADfRvetAyxsakekx.} & \text{yADfR dUradalilx hiDiVtu.}\\[3pt]
& 50 & =200
\end{tabular}

\begin{tabular}{>{$}c<{$}>{$}c<{$}>{$}c<{$}>{$}l<{$}}
& \text{yADiRge} & \text{minUyxTf} & \text{yADiRge}\\[3pt]
\text{athavA} & 8\quad : & 1 \quad :: & 200=25\; \text{minUyxTiya hotitxnalilx hiDiVtu.}\\
\end{tabular}

\item $2$ rUpAyige $3$ maNa belalxvU $3$ rUpAyige $2$ maNa sakakxreyU mArutitxralAgi, obabxnu $99$ rUpAgaLanunx koTuTx A eraDanUnx kUDA tegadukoLaLx beVkeMtalU, avugaLalilx belalxda eraDaSuTx sakakxre ira beVkeMtalU apeVkiSxsutAtxne. hAgAdare, avanu yAvAyxvadanunx eSeTxSuTx tegadu koLaLx beVku?

\begin{equation*}
\left.
\begin{tabular}{>{$}c<{$}>{$}c<{$}>{$}l<{$}}
\text{maNa belalxkekx}& \text{rU.} & \text{maNakekx}\\
3 & :\quad  2 \quad:: & 1=\tfrac{2}{3}\; \text{rU.} \\
\text{maNa sakakxrige}& \text{rU.} & \text{maNakekx}\\
2   &:\quad 3 \quad:: & 2=3\; \text{rU.} \\
\end{tabular}
\right \}
\text{oTuTx}\; 3\tfrac{2}{3}\; \text{rUpAyigaLAdavu.}
\end{equation*}

\begin{equation*}
\left.
\begin{tabular}{>{$}c<{$}>{$}c<{$}>{$}c<{$}>{$}l<{$}}
& \text{rUpAyige}& \text{maNa belAlx} & \text{rU.}\\
\text{Agalu,} &\quad 3\tfrac{2}{3}\qq : & \quad1\qq :: & 99=27\; \text{maNa belalxvu,} \\
& \text{rU.ge}& \text{maNa sakakxre} & \text{rU.ge}\\
& \quad3\tfrac{2}{3}\qq : &\quad 2\qq  :: & 99=54\; \text{maNa sakakxre} \\
\end{tabular}
\right \}
\text{utatxravu.}
\end{equation*}

\item maNa $1$ kekx $2$ rUpAyi $12$ ANeya parxkArakekx tegadukoMDu, maNa $1$ kekx $3$ rU. $5$ ANe $4$ peY parxkArakekx mAridare, sheVkaDeV eSuTx lABa shikukxvadu?

\begin{tabular}{>{$}c<{$}>{$}c<{$}>{$}c<{$}>{$}c<{$}>{$}c<{$}>{$}l<{$}}
&&\text{rU.} & \text{A.} & \text{peY.}\\
\text{ililx} & \text{maNa}\; 1\; \text{kekx} & 3 & 5 & 4 & \text{parxkAra mAridudx.}\\
 & \text{maNa}\; 1\; \text{kekx} & 2 & 12 & 0 & \text{idu avana mUla karxyavu.}\\[1pt]
\cline{2-5}
 & \text{maNa}\; 1\; \text{kekx} & 0 &  9 & 4 & \text{idu baMda lABavu.}\\
\end{tabular}\\

\begin{tabular}{>{$}c<{$}>{$}c<{$}>{$}c<{$}>{$}c<{$}>{$}c<{$}>{$}c<{$}>{$}c<{$}>{$}c<{$}>{$}c<{$}}
\text{hAgAdare}, & \text{rU}. & \text{A.} &  \text{A}. & \text{kA. lABa} & \text{rU}. & \text{rU}. & \text{A.}& \text{peY.}\\[5pt]
& 2 & \multicolumn{2}{c}{$12 : 9$} & 4 & \multicolumn{2}{c}{:: $100=22$} & 3 & 4\tfrac{8}{11}
\end{tabular}

\item maNa $1$kekx $5$ rUpAyi $8$ ANe parxkArakekx tegadukoMDa sarakanunx sheVkaDA $5$ rUpAyi $4$ ANegaLa parxkArakekx naSaTxvAgi mAra beVkAyitu. Agalu maNa $1$kekx eSuTx karxya bidadx hAgAyitu?


\qq\begin{tabular}{>{$}c<{$}>{$}c<{$}>{$}c<{$}>{$}l<{$}}
100 & 0 & 0 & \text{mAra takakxdanunx}\\
\quad5 & 4 & 0 & \text{naSaTxdiMda mAridare,}\\
\cline{1-3}
94 & 12 & 0 & \text{iSuTx rUpAyige mArida hAgAyitu.}
\end{tabular}

\begin{tabular}{>{$}l<{$}>{$}c<{$}>{$}c<{$}>{$}c<{$}>{$}c<{$}>{$}c<{$}>{$}c<{$}>{$}c<{$}>{$}c<{$}}
\text{hAgAdare.}& & & & & & & & \\
\text{rU.} & \text{rU.} & \text{A} & \text{rU.} & \text{A.} & \text{rU.} & \text{A.} & \text{peY.}\\[-10pt]
&&&&&&&&\text{utatxravu.}\\
100 : & 5 & \multicolumn{2}{c}{$8 :: 94$} & 12= & 5 & 3 & 4\tfrac{14}{25}\\[-10pt]
\end{tabular}\\[10pt]

\item $120$ gaja vasatxrXvanunx $735$ rUpAyige mAridare sheVkaDA $5$ rUpAyi lABa shikukxvadAgiyitutx. Adare gaja $1$ kekx mUla karxyaveSuTx?

ililx vasatxrXvanunx $105$ rUpAyinaMte mAruvadakekx $100$ rUpAyi mUla karxvAda hAgAyitaSeTx.

\qq\begin{tabular}{>{$}c<{$}>{$}c<{$}>{$}c<{$}>{$}c<{$}>{$}c<{$}}
\text{hAgAdare,} & &&&\\
& \text{rU.} &\text{rU. karx.} & \text{rU. ge} &  \text{rU. asalu karxya.}\\[2pt]
& 105 : & 100 & :: 735 & =700\\
\end{tabular}

\begin{tabular}{>{$}l<{$}>{$}c<{$}>{$}c<{$}>{$}c<{$}>{$}c<{$}>{$}c<{$}>{$}c<{$}}\text{gajakekx} & \text{rU. karx.} & \text{gajakekx} & \text{rU.} & \text{A.} & \text{peY.} &\\[-6pt]
&&&&&&\text{karxyavu. utatxravu.}\\[-6pt]
\multicolumn{2}{c}{$120 : 700 ::$} & \multicolumn{2}{c}{$1 = 5$} & 13 & 4\\
\end{tabular}\\

\item dara gaja $1$kekx ANe parxkArakekx mAra beVkeMdeNisikoMDare, sheVkaDA $4$ rUpAyi naSaTxvAgutetxMbuvadanunx tiLadu sheVkaDAge $10$ rUpAyi lABa baruva hAge mAra beVkeMdiCeCxYsutAtxne. Agalu avanu dara gajakekx yAva karxyadiMda mAra beVku.

ililx modalaneV karxyada parxkAra $100$ rUpAyi karxyavAguva vasatxrXvanunx $96$ rUpAyige mArida hAgAguvadadxriMda,

\qq\begin{tabular}{>{$}c<{$}>{$}c<{$}>{$}c<{$}>{$}c<{$}}
\text{rU.} & \text{rU. karxya} & \text{ANege}\\[2pt]
96\;\text{kekx} : & \multicolumn{2}{c}{$100\quad ::\quad 8$} & =8\tfrac{1}{3}\; \text{ANe karxyavAyitu.}\\
\end{tabular}\\

$2$neV karxyada parxkAra $100$ rUpAyi karxyavAguva vasatxrXvanunx $110$ rUpAyige koTaTx hAgAguvadadxriMda,

\begin{center}
\begin{tabular}{>{$}c<{$}>{$}c<{$}>{$}c<{$}>{$}c<{$}>{$}c<{$}>{$}c<{$}}
\text{rU.}& \text{rU.}& \text{ANege}& \text{ANe}& \text{peY}\\[-6pt]
&&&&&\text{utatxravu.}\\[-6pt]
\multicolumn{2}{c}{$100\; :\; 110\; ::$} & \multicolumn{2}{c}{$8\tfrac{1}{3} = 9$} & 2\\ 
\end{tabular}\\
\end{center}

\item akAranigU ikAranigU madhayxdalilx $210$ meYlfgaLuMTu. avaribabxru obabxranonxbabxru noVDa takakxdadxkAkxgi Eka kAladalilx horaTu baruvAgeyx $7$ divasakekx parasapxra saMdhisidaru. Aga noVDalAgi akAranU ikAragiMtalU parxti divasavU $6$ meYlfgaLaMte hecAcxgi naDadu baMdiruvadAgi tiLiya baMtu. Agalu avaribabxrU parxti divasavU parxyANa mADidaMthA meYlfgaLeSeTxSuTx?

$
\left.
\begin{tabular}{>{$}c<{$}}
210\; \text{meYlf muMce idadxdUravu.}\\[5pt]
\quad\;42\; \text{akAranu hecAcxgi naDada meYlf.}\\
\end{tabular}
\right \}
$
\begin{tabular}{>{$}l<{$}>{$}l<{$}>{$}l<{$}>{$}c<{$}}
\text{akAranu.} & & & \\[-3pt]
\text{di.} & \text{meYlf hecucx} & \text{di.} & \text{meYlf hecAcxgi}\\
1 : & \quad6\qq ::  &  \multicolumn{2}{c}{$ 7 = 42$\;\text{naDadadudx.}}
\end{tabular}\\[-8pt]

\begin{tabular}{>{$}c<{$}>{$}c<{$}}
\cline{1-1}
2)\; 168\; \text{iSuTx} & \text{meYlf avaribabxrU sheVri naDada hAge Ayitu.}\\
\cline{1-1}\\[1pt]
\end{tabular}

 \quad $84$ idu obobxbabx naDadaMte tiLakoLoLxVNa. Agalu ivanigiMtalU $42$ meYlf akAranu naDadiruvadadxriMda, Ageyx $84+42=126$ meYlf akAranU, matutx $84$  meYlf ikAranU $7$ divasagaLige naDada hAgAyitu. \quad AdadxriMda,


\begin{tabular}{>{$}l<{$}>{$}c<{$}>{$}c<{$}>{$}l<{$}}
\text{di.} & \text{meYlf} & \text{di.} & \text{meYlf}\\
7 & :\; 84\; :: & 1 & =12\;\text{idu ikAranu dara divasavU naDadadudx.}\\
7 & : 126 :: & 1 & =18\;\text{idu akAranu dara divasavU naDadadudx.}
\end{tabular}\\[10pt]

\item obabx gaqhasathxnu tanige $12$ rUpAyi saMbaLaviruva kAladalilx $16$ rU pAyigaLanunx KacuR mADuvanu. $18$ rUpAyi saMbaLavAda meVle $10$ rUpAyigaLanunx KacuR mADutAtx baMdanu. hiVge $2$ vaSaRvAda meVle $96$ rUpAyi shilukx uLiVtu. Adare avanu yAvAyxva saMbaLadiMda eSeTxSuTx divasa kelasa mADira bahudu.

idaralilx avanu saMpAdisidedxlAlx $18$ rUpAyi saMbaLavAda kAladalilx. AdadxriMda,


\begin{tabular}{>{$}c<{$}>{$}l<{$}>{$}c<{$}>{$}c<{$}>{$}l<{$}}
\text{tiMgaLige} & \text{rU.} & \text{saMpAdane} & \text{tiMgaLige} & \text{rU}\\
1 & : & 8 \quad:: & 24 \quad= & 192\; \text{saMpAdaneyU, idaralilx}\\ 
&&&& \;\;96\; \text{rUpAyi shilukx uLadadudx.}  \\
\end{tabular}

$
\left.
\begin{tabular}{>{$}c<{$}>{$}c<{$}>{$}l<{$}}
\text{rU.ge} & \text{tiM} & \text{rU.ge}\\
12 &: 1 : & 96=8\; \text{tiMgaLu}\\
&&\qq16\; \text{tiMgaLu}
\end{tabular}
\right \}
\begin{tabular}{>{$}c<{$}>{$}c<{$}}
96\; \text{rUpAyi sAlakekx vajA}\\
\text{AgabeVku.}
\end{tabular}
$

\item $480$ maNa niVru tuMbuvaMthA oMdu pAterxyuMTu. adaralilx niVranunx tuMbuvadakoVsakxra oMdu saMNa pAterxyanunx matotxMdu doDaDx pAterxyanunx saMpAdisidadxru. adaralilx saMNa pAterx inonxbabxnu tegadu koMDu $4$ minUyxTinalilx $2$ salavU, matotxbabxnu doDaDxpAterxyanunx tegadukoMDu $4$ minUyxTinalilx $3$ salavU tuMbuvadakekx pArxraMBisidarU. AdAgUyx saMNa pAterx eSuTx niVru hiDiyuvadoV adara eraDaSuTx niVru doDaDx pAterx hiDiyutitutx. hiVgiralAgi $12$ minUyxTfnalilx A pAterxyU saMpUNaRvAyitu. Adare A pAterxgaLu eSeTxSuTx maNa niVru hiDiya takakxvugaLAgidadxvu, heVLu?

\begin{tabular}{>{$}c<{$}>{$}c<{$}>{$}c<{$}>{$}l<{$}}
\text{minUyxTige} & \text{satiR} & \text{minUyxTige} & \text{satiR}\\
4 & :\; 2\; :: & 12 & =6\; \text{idu saMNa pAterxyiMda hAkida satiRyU.} 
\end{tabular}


\begin{tabular}{>{$}c<{$}>{$}c<{$}>{$}c<{$}>{$}l<{$}}
\text{minUyxTige} & \text{satiR} & \text{minUyxTige} & \text{satiR}\\
4 & :\; 3\; :: & 12 & =9\; \text{idu doDaDx pAterxyiMda hAkidaMthA satiRyu.} 
\end{tabular}

Agalu oMdoMdu satiRge cikakxdaralilx $1$ maNavU doDaDxdaralilx $2$ maNavU niVru bitetxMdu BAvisoVNa. hAgAdare,

\begin{tabular}{>{$}c<{$}>{$}c<{$}>{$}c<{$}>{$}l<{$}}
\text{salakekx} & \text{maNa saMNa pAterx} & \text{satiRge}\\
1 & :\qq 1\qq :: & 6= & 6\; \text{maNa niVru saMNa pAterxdAyitu.}\\
\text{salakekx} & \text{maNa doDaDx pAterx} & \text{satiRge}  \\
1 & :\qq 2 \qq:: & 9= & 18\; \text{maNa niVru doDaDx pAterxdAyitu.}\\[-5pt]
&&&$---------------$\\[-5pt]
&&& 24\; \text{maNa \;2\; pAterxgaLiMda AdahAgAyitu.}\\
\end{tabular}

\begin{tabular}{>{$}c<{$}>{$}c<{$}>{$}l<{$}}
\text{AdadxriMda, maNakekx} & \text{maNa niVrina saMNa pAterx} & \text{maNakekx}\\
\qq24 & :\qq\quad 1 \qq\quad :: & 480=20\; \text{maNa hiDiya}\\ 
&& \text{takakx parxmANa saMNapAterxdAyitu.}
\end{tabular}\\[10pt]

\begin{tabular}{>{$}c<{$}>{$}c<{$}>{$}l<{$}}
\text{maNakekx} & \text{maNa doDaDx pAterx niVru} & \text{maNakekx}\\
24   &: \qq 2 \qq\qq:: & 480=40\; \text{maNa hiDiya takakxdudx}\\ 
&&\text{doDaDxpAterxV parxmANavAyitu.}
\end{tabular}

\item obabxnu AkAra, ikAra, ukArareMba $3$ janagaLige tananx darxvayxvanunx haMci koTaTxnu. hAyxgeMdare, akAranige $2$ rUpAyi Adare ikAranige $4$ rUpAyigaLu bara beVku. matutx ikAranige $6$ rUpAyi shikikxdare ukAranige $9$ shikakx beVku. hiVge haMcalAgi ikArana pAlige $400$ rUpAyigaLu dorakidavu. Agalu akAra ukAraru pAlina darxvayxgaLeSuTx? matUtx elAlx darxvayxveSuTx?

\begin{tabular}{>{$}c<{$}>{$}c<{$}>{$}c<{$}>{$}l<{$}>{$}l<{$}}
\text{ikAranige baMdare} & \text{akAranige} & \text{ikAranige}\\
4 & :\quad 2 \qq :: & 400 & = 200\; \text{AkArana pAlu.}\\
\text{ikAranige baMdare} & \text{ukAranige} & \text{ikAranige}\\
6 & :\qq 9 \qq :: & 400 & =600\;  \text{ukArana pAlu.}\\
&&& \quad\;400\; \text{ikArana pAlu.}\\[-5pt]
&&&\quad$------$\\[-5pt]
&&& \quad1200\; \text{oTuTx rUpAyi.}\\[-5pt]
&&&\quad$------$\\[-5pt]
\end{tabular}

\item oMdu kelasavanunx $3$ jana gaMDasaru athavA $5$ jana heMgasaru athavA $7$ jana huDugaru $1$ divasadalilx mADutAtxre, hAgAdare A kelasavanunx $1$ gaMDasU $2$ jana heMgasarU $3$ jana huDugarU oTiTxge sheVri eSuTx kAladlilx mADuvaru?
\begin{verse}
idaralilx $1$ divasakekx obabx gaMDasu $\tfrac{1}{3}$ BAga kelasanUnx obabx heMgasu $\tfrac{1}{5}$ BAga kelasavanUnx obabx huDuga $\tfrac{1}{7}$ BAga
kelasavanUnx mADida hAgAguvadadxriMda $\tfrac{1}{3}+\tfrac{2}{5}+\tfrac{3}{7}=\tfrac{122}{105}$ BAgA kelasavanunx
 obabx gaMDasU $2$ jana heMgasarU, $3$ jana huDugarU saha sheVri mADida
hAgAyitu. Adare avarige pUrA kelasa mADuvadakekx $\tfrac{105}{122}$ divasa 
utatxravu.
\end{verse}

\item akAra, ikArareMbuvaribabxrU $1072$ yADfR paridhiyuLaLx oMdu ja\char'371\char'301jina vatuRLa mAgaRdalilx eduru badarige niMtitxdadxru. avaralilx akAranu $1$ minUyxTinalilx $11$ yADfRnunx naDiya takakxvanAgiyU, ikAranu $3$ minUyxTinalilx $34$ yADfR naDiya takakxvanAgiyU idudx avaribabxru parxdakaSxNeyanunx mADuvadakekx tamanx tamamx vAma BAgavanunx kuritu horaTare eSuTx parxdakaSxNegaLanunx mADuvadaroLage avaribabxrU oMdAgi sheVrutAtxre heVLu?

\begin{tabular}{>{$}c<{$}>{$}c<{$}>{$}c<{$}}
\text{ikAranu} & 3\; \text{minUyxTige}\; 34 & \text{yADfR naDiyutAtxne.}\\
\text{akAranu} & 3\; \text{minUyxTige}\; 33 & \text{yADfR naDiyutAtxne.}\\
\cline{2-2}\\[-12pt]
& 3\; \text{minUyxTige}\; 1 & \text{vetAyxsavAyitu. adhaR}
\end{tabular}

\begin{tabular}{>{$}c<{$}>{$}c<{$}>{$}c<{$}}
\text{yADfR vetAyxsakekx.} & \text{yADfR naDiyutAtxne.} & \text{yADf vetAyxsakekx yADfR}\\
1 & :\qq 34 \qq :: & 536=18224\\
& \multicolumn{2}{c}{\qq\qq\text{idu avaribabxrigU saMdhisida dUravu.}}
\end{tabular}

\begin{tabular}{>{$}c<{$}>{$}c<{$}>{$}l<{$}}
\text{Agalu}, & \text{yADfRge. parxdakaSxNe} & \text{yADfRge}\\
& 1072 \quad :\quad 1\quad :: &18224=17\; \text{parxdakiSxNegaLu utatxra.}\\
\end{tabular}

\item obabxnu $5$ divasakekx $4$ rUpAyigaLanunx saMpAdisutAtxne athavA $4$ sholxVkagaLanunx kaliyutAtxne matutx $3$ divasakekx $4$ rUpAyigaLanunx saMpAdisutAtxne athavA $4$ sholxVkagaLanunx kaliyutAtxne. matutx $3$ divasakekx $2$ rUpAyigaLanunx vecacx mADutAtxne athavA $2$ sholxkagaLanunx mariyutAtxne. hAgAdare, $1$ vaSaRkekx eSuTx rUpAyigaLanunx saMpAdishAyxnu? matUtx eSuTx sholxkagaLanunx kalitAnu, heVLu?

\begin{tabular}{>{$}l<{$}>{$}l<{$}>{$}l<{$}>{$}l<{$}}
& \text{di.} & \text{rU. athavA sholx.} & \text{di.}\\
& 5 \quad: & 4 \qq\qq:: & 1=\tfrac{4}{5}\; \text{rUpAyanAnxgali sholxVkagaLanAnxgali saMpAdisutAtxne.}\\

& \text{di.} & \text{rU. athavA sholx.} & \text{di.}\\
& 3\quad : & 2 \qq\qq :: & 1=\tfrac{2}{3}\; \text{rUpAyiyanAnxgali sholxkagaLanAnxgali kaLakoLuLxtAtxne.}\\[10pt]
\text{Agalu} &\multicolumn{2}{c}{$ \dfrac{4}{5}-\dfrac{2}{3}=\dfrac{12-10}{15}=\dfrac{2}{15}$}&  \text{rUpAyigaLanAnxgali sholxkagaLanAnxgali}\; 1\; \text{divasakekx}\\[10pt]
&&&\text{saMpAdisada hAgAyitu.}\\[10pt]
& \text{di.} & \text{rU. athavA sholx.} & \text{di.}\\[10pt]

\text{Adare} & \multicolumn{2}{c}{$\dfrac{1}{1}\quad : \dfrac{2}{15}\quad :: \dfrac{360}{1}=$} & 48\;\text{rUpAyi athavA}\; 48\; \text{sholxVkavanunx saMpAdisutAtxne.}
\end{tabular}

\item obabx vataRkanu $4$ rUpAyige $5$ maNadaMte tegadukoMDu $3$ rUpAyige $2$ maNadaMte mAri biDalAgi, $150$ rUpAyi lABa baMtu. Adare avanalilxdadx baMDavALa eSuTx?

idaralilx avana asalu $1$ rUpAyi itetxMdu BAvisoVNa.

\begin{tabular}{>{$}c<{$}>{$}c<{$}>{$}c<{$}>{$}l<{$}}
\text{Agalu,} & \text{rU.ge} & \text{maNa} & \text{rU. ge}\\
& 4 \quad : & 5\quad :: & 1=\tfrac{5}{4}\; \text{maNavanunx avanu tegadukoMDadu.}
\end{tabular}


\begin{tabular}{>{$}c<{$}>{$}c<{$}>{$}l<{$}}
\text{maNakekx} & \text{rU} & \text{maNakekx}\\
\quad\dfrac{2}{1} \quad: & \quad\dfrac{3}{1} \quad:: & \quad\dfrac{5}{4}=\dfrac{15}{8}=1\dfrac{7}{8}\; \text{rUpAyi mAralAgi baMdadudx.}\\[10pt]
&& \qq 1\; \text{avana asalu, jAtA bAki nape uLidadudx}\; \tfrac{7}{8}\;
\end{tabular}


\begin{tabular}{>{$}c<{$}>{$}c<{$}>{$}c<{$}>{$}c<{$}>{$}l<{$}}
\text{Agalu rU.} & \text{naPege.} \text{rU.} & \text{asalu.} & \text{rU. naPege.}\\
& \dfrac{7}{8} & \dfrac{1}{1} & \dfrac{150}{1} & =171\dfrac{3}{7}\; \text{rUpAyi asalu utatxravu.} 
\end{tabular}

\item $240$ rUpAyigaLanunx $3$ janagaLige haMcikoMDu. hAyxgeMdare, obabxnige $1$ rUpAyi baMdare matotxbabxnige $2$ rUpAyigalU inonxbabxnige $3$ rUpAyigaLU baMda hAge sariyAgira beVku. Agalu yArAyxrige eSeSuTx?

idaralilx $1+2+3=6$ idu mUru janariMda Adadudx. Agalu,

\begin{tabular}{>{$}l<{$}>{$}l<{$}>{$}l<{$}>{$}l<{$}>{$}c<{$}>{$}l<{$}}
&\text{rU}& \text{rU. na} & \text{BAgakekx.} & \text{rU.}\\
6\; \text{rU. ge} \;:&240 &:: & 1& =\; 40& \text{oMdaneVyavanige}\\
6\; \qq : &240 & :: & 2 & =\; 80 &\text{eraDaneVyavanige}\\
6\; \qq : & 240 & :: & 3 & \;=\; 120 &  \text{mUraneVyavanige}\\ 
\cline{5-5}
&&&& \text{oTuTx}. 240\\
\cline{5-5}
\end{tabular}

\item 
oMdu seYnayxdalilx $5$ kaMpenigaLidadxvu. modalaneVdaralilx $54$ janavU eraDaneVdaralilx $51$ janavU mUraneVdaralilx $48$ janavU nAlukxneVdaralilx $39$ janavU $5$neV kaMpeniyalilx $36$ janavU hiVge irutitxdadxru. avaralilx oMdu koVTeya kAvaligoVsakxra $1$ JAmakekx $76$ janagaLa parxkArakekx kaLuhisa beVkAgidadxre, yAvAyxva kaMpenigaLiMda eSeTxSuTx janagaLanunx kaLuhisa beVku?

\qq $54+51+48+39+36=228$ idu oTuTx janagaLU.

\begin{tabular}{>{$}l<{$}>{$}c<{$}>{$}c<{$}>{$}c<{$}>{$}l<{$}>{$}l<{$}}
\text{Agalu}, & \text{janagaLalilx}. & \text{janagaLanunx} & \text{kaLuhisa beVku.} & \text{janaru.}\\
& 228 & : & 76 &::\; 54=18 & \text{janaru.}\\
& 228 & : & 76 &::\;51=17\\
& 228 & : & 76 &::\; 48=16\\
& 228 & : & 76 &::\; 39=13\\
& 228 & : & 76 &::\; 36=12\\
\cline{5-5} 
& & & & \text{oTuTx}\;76\\
\cline{5-5} 
\end{tabular}

\item kakAra gakArareMbuvaribabxru vAyxpArakoVsakxra saMmatipaTuTx, avaralilx kakAranu $500$ rUpAyigaLanunx koTuTx $4$ tiMgaLu, matutx gakAraneMbuvanu $600$ rUpAyigaLanunx koTuTx $5$ tiMgaLU AgiralAgi, muMde avarige $240$ rUpAyigaLanunx koTuTx $5$ tiMgaLU AgiralAgi, muMde avarige $240$ rUpAyi lABa shikikxdare, yArAyxru eSeTxSuTx tegadukoLaLx beVku?

$
\left.
\begin{tabular}{>{$}c<{$}}
500 \times 4 =2000\\
600 \times 5 =3000\\
\qq \text{oTuTx}\; 5000
\end{tabular}
\right \}
$
\begin{tabular}{>{$}c<{$}>{$}l<{$}>{$}l<{$}>{$}l<{$}}
\text{rU. ge} & \text{rU. lABa} & \text{rU. ge}\\
5000 & :\quad 240\quad:: & 2000  =  96 & \text{rU. kakAranige}\\
5000 & :\quad 240 \quad:: & 3000  =  144 & \text{rU. gakAranige}\\
\cline{3-3}
&& \text{oTuTx}\; 240\\
\cline{3-3}
\end{tabular}

\item cakAra, jakAra, TakArareMba mUru janaru $342$ rUpAyige oMdu hululx gAvalanunx gutitxgegx tegadukoMDaru. adaralilx cakArana $7$ AkaLu $3$ tiMgaLU matutx jakArana $9$ AkaLu $5$ tiMgaLU matutx TakArana $4$ AkaLu $1$ vaSaRvU meVdiralAgi, A gutitxgeV haNavanunx yArAyxru eSeTxSuTx koDa beVku, heVLu?

\begin{tabular}{>{$}c<{$}>{$}l<{$}}
7\times 3=21 &\text{AkaLu cakAranadu.}\\
9\times 5=45 &\text{AkaLu jakAranadu.}\\
4\times 12=48 &\text{AkaLu TakAranadu.}\\[-8pt]
\qq\quad\; $-----$\\[-8pt]
\qq\quad\; 144& \text{AkaLu mUru janagaLadU sheVri meVda hAge Ayitu.}
\end{tabular}

\begin{tabular}{>{$}c<{$}>{$}l<{$}>{$}c<{$}>{$}l<{$}>{$}l<{$}}
\text{Agalu,} & \text{AkaLige} & \text{rU.} & \text{AkaLige}\\
& 114 & : 342 :: & 21=63 &\text{rU. cakAra koDa takakxdudx.}\\
& 114 & : 342 :: & 45=135 &\text{rU. jakAra koDa takakxdudx.}\\
& 114 & : 342 :: & 48=144 & \text{rU. TakAra koDa takakxdudx.}\\
\cline{4-4}
&&& \;\text{oTuTx}\; 342 & \text{rUpAyi.}
\end{tabular}

\item obabxnalilx $5$ roTiTxgaLU inonxbabxnalilx $7$ roTiTxgaLU idadxvu. avaru tinanx beVkeMba samayakekx matotxbabxnige yita baMdanu. Aga mUru janagaLU A roTiTxgaLanunx samanAgi tiMdaru. A meVle baMdaMthA A senxVhitanu saMtoVSadiMda $12$ duDuDxgaLanunx koTaTxnu. Adare adu $5$ roTiTxyavanige eSuTx? matutx ELu roTiTxyavanige eSuTx duDuDxgaLu bara beVku heVLu?

\qq $5+7=12\div3=4$\quad \text{roTiTx obobxbabxnu tiMdanu.}


\begin{tabular}{>{$}c<{$}>{$}l<{$}>{$}l<{$}}
\text{Agalu},& 5 & \text{roTiTx}\\
& 4 & \text{tA tiMdA}\\
\cline{2-2}
& 1 & \text{baMdavanige koTATx}\; $+$
\end{tabular}
\begin{tabular}{>{$}c<{$}>{$}l<{$}}
7 & \text{roTiTx}\\
4 & \text{tA tiMdA}\\
\cline{1-1}
\; 3 & \text{baMdavanige koTATx}= 4\quad \text{roTiTx baMdavanu tiMdadudx.}
\end{tabular}

\begin{tabular}{>{$}c<{$}>{$}c<{$}>{$}c<{$}>{$}l<{$}}
\text{roTiTxge.} & \text{duDuDx koTATx.} & \text{roTiTxge.}\\
4 & :\quad  12 \quad::& 1 &=3\; \text{duDuDx idu\; $5$\; roTiTxyavanige.}\\
4 & :\quad 12 \quad ::& 3& =9\; \text{duDuDx idu\; $7$\; roTiTxyavanige.}
\end{tabular}

\item obabxnu mAvina haMNugaLanunx ANege $4$ra meVrige tegadukoMDanu. A meVle A haMNugaLalilx $\tfrac{1}{3}$ BAgavanunx $8$ haMNige $10$ duDiDxna hAge mAridanu. A meVle uLida haMNugaLanenxlAlx oTiTxge $2\tfrac{1}{2}$ rUpAyige koTuTx biTaTxnu. idariMda avanige $13$ ANe $4$ peY naPe baMtu. Adare haMNugaLeSuTx?

$1$ ANege $4$ haMNu Adare $1$ haMNu\;=\;$\dfrac{3}{4}$\; duDuDx.\\
\begin{tabular}{>{$}c<{$}>{$}c<{$}>{$}c<{$}>{$}l<{$}}
\text{haMNige} & \text{duDuDx} & \text{haMNige}\\
8 & : 10 :: & \tfrac{1}{3}& =\tfrac{5}{12}\; \text{duDuDx.}
\end{tabular}

$\dfrac{3}{4}-\dfrac{5}{12}=\dfrac{9-5}{12}=\dfrac{4}{12}=\dfrac{1}{3}$\quad \text{duDuDx sheVSa.}\\

$
\left.
\begin{tabular}{>{$}c<{$}>{$}c<{$}>{$}c<{$}>{$}l<{$}}
& 2\tfrac{1}{2}\; \text{rU}=& 120 & \text{duDuDx.}\\
13 & \text{A.} 4\; \text{peY.}= &\; 40 & \text{duDuDx naPe.}\\
\cline{3-3}
&& \; 80 & \text{duDuDx.}
\end{tabular}
\right \}
$
\begin{tabular}{>{$}c<{$}>{$}c<{$}>{$}c<{$}>{$}l<{$}}
\text{duDuDx} & \text{sheVSakekx} & \text{haMNu} & \text{ke.}\\[2pt]
\tfrac{1}{3} & \qq :& \quad\tfrac{1}{1} \; :: & 80\\[2pt]
&& =240 & \text{haNuNx.}
\end{tabular}

\item oMdu nAramx\char'366 sUkxlfnalilx $100$ jana vidAyxthiRgaLidadxru. avaralilx kelavarige $7$ rU. meVrigU kelavarige $5$ rU. meVrigU saMbaLagaLu salulxtAyitutx. tiMgaLu tuMbida meVle oTuTx saMbaLa $684$ rUpAyigaLu dorakidavu. adaralilx mAyxsaTxrf saMbaLa $40$ rUpAyigaLAgidadxre, A sUkxlfnalilx yAvAyxva saMbaLavanunx tegadukoLaLx takakxvaru, eSeTxSuTx huDugaridadxru heVLu?

\begin{tabular}{>{$}c<{$}>{$}c<{$}>{$}c<{$}>{$}c<{$}>{$}c<{$}>{$}l<{$}}
\text{oTuTx jana}\\
\qq 100\times7 & \text{hecicxna dara saMbaLa}=&700&&684& \text{rU.}\\
&& 644 & \text{rU. baMdadudx.} & \;40 & \text{mAsaTxradu.}\\
\cline{3-3}&\\[-14pt] \cline{5-5}
\end{tabular}\\[-4pt]

\quad$
\left.
\begin{tabular}{>{$}c<{$}>{$}c<{$}}
7-5=2\; \text{idu hecucx matutx}\\
\quad \text{kaDemx}  \text{daragaLiruva aMtara.}
\end{tabular}
\right \}
$
\begin{tabular}{>{$}c<{$}>{$}c<{$}>{$}c<{$}}
2) &\; 56 & \qq\qq644 \quad\; \text{rU.}\\
& \;
28 & \text{ivaru $5$ rU. darada huDugaru.}
\end{tabular}\\[-4pt]

\begin{tabular}{>{$}l<{$}>{$}c<{$}>{$}l<{$}}
\text{athavA}  & 644 & \text{rU. ba.}\\
100 \times 5\quad  \text{kaDemxdara}\quad=& 500 &\qq 72\; \text{dara\; $7$\; rUpAyina vidAyxthiRgaLU.}\\
\cline{2-2}
\qq\text{darada aMtara}\quad 2) &144\\
\cline{2-2}
& \quad72 & \text{dara\; 7\; rU.navaru}\\
& \quad28 & \text{dara\; 5\quad ,,}
\end{tabular}
\end{enumerate} 


