\newpage

\begin{center}
{\bf\LARGE 122neV parxkaraNa.}\\
\vskip .3cm
{\bf\large saMkarxmaNa leKaKxgaLanunx kuritadudx.}\\
\end{center}

saMkarxmaNaveMdare, bija gaNitadiMda sAdhayxvAga takakx kelavu samIkaraNa rUpAvAda leKaKxgaLanunx mADuva vidhA\-navU. avugaLalige sUtarxgaLanunx kalipxsi bariyalapxTiTxrute. AyA jAtiV leKaKxgaLanunx hAyxge heVLidAgUyx keVLi, adara dara sUtarxgaLa parxkArakekx mADa bahudAgirutatxve.

\begin{center}
{\bf 1neV parxkAra.}
\end{center}

udahAraNe, eraDu saMKayxgaLa yoVgavu $101$ matutx avugaLa aMtaravu $25$ Agutatxve, Adare averaDu saMKayxgaLAyxvu.

\begin{center}
{\bf\large sUtarx.}
\end{center}

\begin{verse}
kaM|| yoVgadoLaMtara kaLiyuta | lAgadhisaloMdu saMKayxvanU|| yoVga saMKayxdoLu kUDi\-se| beVgadoLinonxMdu saMKayx baruvadu gaNakA||

vi|| yoVga saMKayxdalilx aMtara saMKayxvanunx kaLadu athiRsidare oMdu saMKayx baruvadu adanunx A yoVga saMKayxdalilx kaLadare, matotxMdu saMKayx baruvadu.
\end{verse}

\begin{tabular}{ll>{$}c<{$}l}
riVti, & yoVga & 101\\
& aMtara & 25\\
&& 76\\[-6pt]
& adhiRsalu & $-----$\\[-6pt]
&& 38 & idu oMdu saMKavu.
\end{tabular}
\begin{tabular}{>{$}c<{$}l}
101\\
38\\
\cline{1-1}
63 & idu matotxMdu saKayxyu.
\end{tabular}

\begin{center}
{\bf\large 132neV aBayx udAharaNe.}
\end{center}

\begin{enumerate}[\rm(1)]
\item eraDu saMKayxgaLa yoVgagaLu $44, 56, 76, 93,$ matUtx avugaLa aMtaragaLu $2, 6, 8, 13,$ Adare, A saMKayxgaLAvu?

\item eraDu saMKayxgaLa yoVgagaLu $16\tfrac{3}{4}, 17\tfrac{5}{8}, 22\tfrac{1}{16}, 20\tfrac{5}{16},$ matutx avugaLa aMtaragaLu $4\tfrac{1}{4}, 6\tfrac{7}{8}, 3\tfrac{7}{16}, 4\tfrac{1}{16},$ Aguvavu. Agalu, A saMKayxgaLAyxvu? 
\end{enumerate}

\begin{center}
{\bf 2neV parxkAra.}
\end{center}

udahAraNe, eraDu saMKayxgaLa guNAkAra $51$ avugaLa aMtara $14,$ A eraDu saMKayxgaLAyxvu.

\newpage

\begin{center}
{\bf\large sUtarx.}
\end{center}

\begin{verse}
kaM|| aMtaradogaRdi labadhxma| deMtihadadanAloLiridu kUDuta mUlaM||
eMtAguvadadaroLaMtara | saMtasadiM kUDi kaLadu adhiRse samaneY||

vi|| aMtara saMKayxda vagaRdalilx guNAkAra labadhxda catuguRNavanunx kUDisi A oTiTxna vagaR mUlavanunx tegadu adanunx divxsAthxpisi adaralilx aMtara saMKayxvanunx kUDi, kaLadu adhiRsidare, guMNayx guNakAMKigaLaguvavu.
\end{verse}

\begin{tabular}{lr>{$}l<{$}>{$}c<{$}>{$}c<{$}>{$}l<{$}>{$}l<{$}}
riVti, & aMtara saMKeyx & 14 && \text{rAshi} & 5\tfrac{1}{4}\\[-4pt]
& vagiRsalu & $---$ &&& $---$\\[-4pt]
&& 196 & + && 204& =\sqrt{400}=20\\
& Agalu& \\ \cline{3-6}\\
&& 20\;\text{mUka} &&& 20& \text{mUla}\\
&& 14\; \text{aMtara} &&& 14 & \text{aMtara}\\
& kUDi kaLiyalu& \\ \cline{3-6}\\
&& 34 && ~~6\\
& adhiRsalu& \\ \cline{3-6}\\

&& 17 && ~~3 && \text{iveV utatxragaLU.}
\end{tabular}

\begin{center}
{\bf\large 133neV aBayx udAharaNe.}
\end{center}

\begin{enumerate}[\rm(1)]
\item eraDu saMKayxgaLa guNAkAragaLu $30, 52, 70, 96$ matutx avugaLa aMtaragaLu $7, 9, 9, 10$ iruvavu, A saMKayxgaLAyxvu?

\item eraDu saMKayxgaLa guNAkAra $303\tfrac{3}{4}, 133, 80$ matutx avugaLa aMtaravu $5\tfrac{1}{4}, 12, 11$ Aguvavu, A saMKayxgaLAyxvu?
\end{enumerate}

\begin{center}
{\bf 3neV parxkAra.}
\end{center}

udAharaNe, eraDu saMKayxgaLa gunAkAra $96$ avugaLa yoVga $20$ Adare, A eraDu saMKayxgaLAyxvu.

\begin{center}
{\bf\large sUtarx}
\end{center}

\begin{verse}
kaM|| yoVgavanadhiRsi vagiRsu | tAgadaroLaLxbadxvaLidu mUlava tegadada|| nUyxgAdhaRdi kUDi kaLiyali| kAgalu barutihudu guMNayx guNakAMkigaLeY||

vi|| yoVga saMKayxvanunx AdhiRsi, adanunx vagiRsi, adaralilx guNAkAra labadhxvanunx kaLadu sheVSada vagaR mUlavanunx tegadu A mUla saMKeyanunx yoVgAthaR saMKeyalilx kUDisidaroMdu saMKayxvu kaLadare matotxMdu saMKayxvu Aguvadu.
\end{verse}

\begin{tabular}{lr>{$}l<{$}l}
riVti. & yoVgasaMKeyx & ~20\\[-4pt]
& adhiRsalu & ~$---$\\[-4pt]
&& ~10\\[-4pt]
& vagiRsalu & ~$---$\\[-4pt]
&& 100\\
& rAshi & ~~96 & kaLiyalu\\
&&~~$---$\\[-4pt]
&& \quad~4 & sheVSa\\
& adara mUla\\
&& \quad~2
\end{tabular}
\begin{tabular}{>{$}c<{$}>{$}l<{$}}
\text{yoVgAdhaRgaLu}\\
10 & 10\\
\qq\quad~2\; \text{kUDisalu} & ~~2\;\text{kaLiyalu}\\[-4pt]
$---$ & $---$\\[-4pt]
12 & ~~8\; \text{utatxragaLu.}
\end{tabular}

\begin{center}
{\bf\large 134neV aBayx udAharaNe.}
\end{center}

\begin{enumerate}[\rm(1)]
\item eraDu saMKayxgaLu guNAkAragaLu $98, 144, 198, 273$ matutx avugaLa yoVgagaLu $21, 25, 29, 34$ Adare, A saMKayxgaLAyxvu?

\item eraDa saMKayxgaLa guNAkAravu $37\tfrac{1}{8}, 57\tfrac{3}{8}, 66\tfrac{5}{8},$ avugaLa yoVgagaLu $12\tfrac{1}{4}, 17\tfrac{1}{4}, 16\tfrac{3}{4},$ Agidadxre, A saMKayxgaLAyxvu?
\end{enumerate}

\begin{center}
{\bf 4neV parxkAra.}
\end{center}

udAharaNe, guMNayx guNakagaLa yoVga $18$ avugaLa BAga labadhx $2$ Adare, A saMKayxgaLAyxvu?

\begin{center}
{\bf\large sUtarx.}
\end{center}

\begin{verse}
kaM|| BAgadi rUpapa sheVrisu | tAgadaroLa gUyxVgasaMkayx BAgise PalavuM|| BAga rUpakava PaladiM| dAgiriyalf guMNayx guNaka samanaM takukxM||

vi|| BAga labadhxdalilx rUpaveMdare, $1$ sheVrisi adariMda yoVga saMKayxvanunx BAgisi baMda PaladiMda BAga labadhxvanUnx rUpaka saMKeyanUnx guNisidare baruva labadhxgaLeV utatxragaLAgiruvavu.
\end{verse}

$
\left.
\begin{tabular}{ll>{$}l<{$}>{$}l<{$}}
riVti, & BAgalabadhx & ~2 & \text{rUpaka}\; 1\\
&& ~1 \\[-4pt]
&& $---$ & \text{yoVga saMKeyx}\\[-4pt]
&&3) & 18\\
\cline{3-4}
&&& ~~6\;\text{Palavu}
\end{tabular}
\right \}
$
\begin{tabular}{l>{$}c<{$}>{$}c<{$}}
&\text{BAga labadhx} & \text{rUpaka saMkeyx}\\
&2 & 1\\
&\qquad6\; \text{Pala} & \qquad6\; \text{Pala}\\[-6pt]
\text{guNisalu}&\\[-6pt] \cline{2-3}
&12 & \qquad~6 \; \text{iveV}\\[3pt]
& \quad\text{guMNayx guNakagaLU}
\end{tabular}

\begin{center}
{\bf\large 135neV aBayx udAharaNe.}
\end{center}

\begin{enumerate}[\rm(1)]
\item eraDu saMKayxgaLa yoVgagaLu $20, 18, 21, 32$ matutx avugaLa BAga labadhxgaLu $3, 2, 2, 3$ Agalu, A saMKayxgaLAyxvu?

eraDu saMKayxgaLa yoVgagaLu $9\tfrac{3}{4}, 16\tfrac{3}{4}, 21\tfrac{1}{4}$ matutx avugaLa BAga labadhxgaLu $2\tfrac{1}{4}, 3\tfrac{3}{16}, 3\tfrac{1}{4},$ Agalu, A saMKayxgaLAyxvu?
\end{enumerate}

\begin{center}
{\bf 5neV parxkAra.}
\end{center}

udAharaNe, eraDu saMKayxgaLa aMtaravu $18$ matutx avugaLa BAga labadhxvu $2$ Agalu, A saMKayxgaLAyxvu?

\begin{center}
{\bf\large sUtarx.}
\end{center}

kaM|| BAgadi rUpava naLiyuta| lAgadariMda daMtaravanu BAgise PalavuM|| BAga rUpAkava PaladiM| dAgiriyalu baruvasaMKayx utatxramakukxM|

vi|| BAga labadhxdalilx rUpakaveMdare, $1$ nunx kaLadu uLidadadxriMda aMtara saMKayxvanunx BAgisi baruva PalAMki\-yiMda BAga labadhxvanUnx rUpakavanUnx guNisalAgi baruva saMKayxgaLeV utatxragaLAgiruvavu.

$
\left.
\begin{tabular}{ll>{$}l<{$}>{$}l<{$}}
riVti. & BAga labadhx & 2 & \text{rUpaka}\\
& & 1 & \quad1\\[-4pt] 
& kaLiyalu & $---$ & \text{aMtara saMKeyx}\\[-4pt]
&& 1)& 18\\[-6pt]
&&&$---$\\[-6pt]
&&& 18\; \text{Palavu.}
\end{tabular}
\right \{
$
\begin{tabular}{>{$}c<{$}>{$}c<{$}}
\text{BAga labadhx} & \text{rUpaka}\\
2 & 1\\
\quad\;18\; \text{Pala} & \quad\;18 \; \text{Pala}\\[-2pt]
\text{guNisalu}&$-----------------$\\
36 & \qq18\; \text{utatxravu}
\end{tabular}

\begin{center}
{\bf 136neV aBayx udAharaNe.}
\end{center}

\begin{enumerate}[\rm(1)]
\item eraDu saMKayxgaLa aMtaragaLu $8, 14, 12, 7,$ matutx avugaLa BAga labadhxgaLu $2, 3, 3, 2$ Agalu, A saMKayxgaLAyxvu?

\item eraDu saMKayxgaLa aMtaragaLu $3\tfrac{3}{4}, 8\tfrac{3}{4}, 11\tfrac{1}{4}$ matutx avugaLa BAga labadhxgaLu $2\tfrac{1}{4}, 3\tfrac{3}{16}, 3\tfrac{1}{4}$ Agalu, A saMKayxgaLAyxvu?
\end{enumerate}

\begin{center}
{\bf 6neV parxkAra.}
\end{center}

udAharaNe, eraDu saMKayxgaLa aMtara $8$ matutx avugaLa vagARMtara $400$ Adare, A eraDu saMKayxgaLAyxvu?

\begin{center}
{\bf\large sUtarx.}
\end{center}

\begin{verse}
kaM|| aMtaradi vagaRdaMtara | veMtiralada BAgisayxda roLaMtara saMKayxva|| naMtara kUDuta kaLadada naMtara diRsemUla saMKayx samanaMtakuM||

vi|| aMtara saMKayxdiMda vagaRdaMtara saMKayxvanunx BAgisi adaroLage aMtara saMKayxvanunx kUDi adhiR\-sidare, oMdu saMKayxvu kaLadu adhiRsidare matotxMdu saMKayxvu baruvadu.
\end{verse}

\begin{tabular}{l>{$}l<{$}>{$}l<{$}>{$}l<{$}>{$}l<{$}>{$}c<{$}l}
riVti. & 400& \div & ~8 & = & 50 \\
&&&&&\qq8\; \text{aMtara}\\
&&&&& \text{kUDikaLiyalu} \\
&&&&& 58\\
&&&&& \text{adhiRsalu}\\
&&&&&29
\end{tabular}
\begin{tabular}{>{$}c<{$}}
50\\
\qq\qq8\; \text{aMtara saMKeyx}\\
\\
~42\\
\\
\qq\qquad21\; \text{utatxravu.}
\end{tabular}

\begin{center}
{\bf\large 137neV aBayx udAharaNe.}
\end{center}

\begin{enumerate}[\rm (1)]
\item eraDu saMKayxgaLa aMtaragaLu $3, 12, 17, 17,$ matutx avugaLa vagARMtaragaLu $81, 288, 561, 799$ Agalu, A saMKeyxgaLAyxvu?

\item eraDu saMKayxgaLu aMtaragaLa $1\tfrac{1}{2}, 3\tfrac{1}{4}, 4\tfrac{1}{2}$ matutx avugaLa vagARMtaragaLu $11\tfrac{1}{4}, 43\tfrac{1}{16}, 74\tfrac{1}{4},$ Agalu, A saMKayxgaLAyxvu?
\end{enumerate}

\begin{center}
{\bf 7neV parxkAra.}
\end{center}

udAharaNe, eraDu saMKayxgaLa yoVga $68$ avugaLa vagARMtara $408$ Agalu, A eraDu saMKayxgaLAyxvu?

\begin{center}
{\bf\large sUtarx.}
\end{center}

\begin{verse}
kaM|| yoVgadoLa gogaR daMtara| BAgisuta danUyxVgadoLage kUDutalUyxnaM| beVgadi mADayxva nadhiRsa| lAgalu barutihavugaNaka mUlAMkigaLu|

vi|| yoVga saMKayxdiMda vagARMtara saMKayxvanunx BAgisi, A Palavanunx yoVga saMKeyxyalilx kUDisi adhiRsidare, oMdu saMKayxvu kaLadu adhiRsidare matotxMdu saMKayxvu baruvadu.
\end{verse}

$
\left.
\begin{aligned}
& \text{riVti.} & 408 \div 68=6\\
\end{aligned}
\right \{ 
\text{yoVga saMKayx}\\[-12pt]
$
\begin{equation*}
\begin{aligned}
\begin{tabular}{r>{$}l<{$}>{$}l<{$}>{$}l<{$}l}
  &~~68 &&  68 \\
  &\quad~6 & \text{Pala} &  ~~6&  Pala\\
\cline{2-4}
kUDi kaLiyalu & ~~74 && ~62\\
adhiRsalu&\\[-4pt] \cline{2-4}
& 37 && 31 & utatxravu.
\end{tabular}
\end{aligned}
\end{equation*}

\begin{center}
{\bf\large 138neV aBayx udAharaNe.}
\end{center}

\begin{enumerate}[\rm(1)]
\item eraDu saMkayxgaLa yoVgagaLu $27, 24, 33, 47$ avugaLa vagARMtaragaLu $81, 288, 561, 799$ Agalu, A saMKayxgaLAyxvu.

\item eraDu saMKayxgaLa yoVgavu $7\tfrac{1}{2}, 13\tfrac{1}{4}, 16\tfrac{1}{2}$ avugaLa vagARMtaragaLu $11\tfrac{1}{4}, 43\tfrac{1}{10}, 74\tfrac{1}{4}$ Agalu, A saMKayxgaLAyxvu?
\end{enumerate}

\begin{center}
{\bf 8neV parxkAra.}
\end{center}

udAharaNe, eraDu saMKayxgaLa yoVga $27$ avugaLa vagaR yoVga $369$ Agalu, A eraDu saMKayxgaLAyxvu.

\newpage

\begin{center}
{\bf\large sUtarx.}
\end{center}

\begin{verse}
kaM|| yoVgada vagaRva nirutiha| yoVgogARda divxguNadoLage kaLiyutatxde 

kaM|| Agalu mUlavategadada| nUyxgadoLaM kUDikaLadu adhiRsesamaneY||

vi|| yoVga saMKayxda vagaRvanunx, vagaR yoVgada divxguNadalilx kaLadu sheVSada vagaR mUlavanunx tegadu adanunx yoVga saMKeyalilx kUDisi adhiRsidare, oMdu saMKayxvU kaLadu adhiRsidare matotxMdu saMKeyU baruvadu.
\end{verse}

\begin{tabular}{l>{$}c<{$}>{$}c<{$}>{$}c<{$}>{$}l<{$}l}
riVti. & \text{yoVga saMKayx} & \text{varagx yoVga} & \text{athavA}\\
& 27 & 369 & \text{yoVga saMKeyx}\\
& \text{vagiRsalu} & \text{divxguNisalu} & 27 & 27\\
& 729 & 738 & \qq3\;\text{mUla} & ~~3 & \text{mUla}\\
& &  727 & \text{kUDi kaLiyalu} & $---$ &\\
&& \text{kaLiyalu} & 30 & 24\\
&& 9 & \text{adhiRsalu} & $---$ \\
&& \text{varagx mUlavu}& 15 & 12\\
&&3 & \text{utatxravu.}
\end{tabular}

\begin{center}
{\bf\large 139neV aBayx udAharaNe.}
\end{center}

\begin{enumerate}[\rm (1)]
\item eraDu saMKayxgaLa yoVgagaLu $15, 18, 23$, $23$ matutx avugaLa varagx yoVgagaLu $113, 180, 277, 289, $ Agalu, A saMKayxgaLAyxvu?

\item eraDu saMKayxgaLa yoVgagaLu $15\tfrac{1}{2}, 18\tfrac{1}{2}, 23\tfrac{3}{4}$ matutx avugaLa varagx yoVgagaLu $121\tfrac{1}{4}, 192\tfrac{9}{16}, 298\tfrac{9}{16}$ Agalu, A saMKayxgaLAyxvu? 
\end{enumerate}

\begin{center}
{\bf 9neV parxkAra.}             
\end{center}

udAharaNe, eraDu saMKayxgaLa aMtara $2$ avugaLa varagx yoVga $52$ Agalu, A saMKayxgaLAyxvu.

\begin{center}
{\bf\large sUtarx.}
\end{center}

\begin{verse}
kaM|| aMtara saMKayxva noVgiRsu| teMtihadA vagaR yoVga divxguNisikaLadada|| keMtu mUlavanu noVDuta | laMtaradali kUDikaLadu adhiRse samaneY||

vi|| aMtara saMKayxda varagxvanunx varagx yoVga saMKaya divxguNadalilx kaLadu sheVSada vagaR mUlavanunx aMtaradalilx kUDisi adhiRsidare, oMdu saMKayxvu kaLadu adhiRsidare matotxMdu saMKayxvU baruvadu.
\end{verse}

\begin{tabular}{l>{$}c<{$}>{$}c<{$}>{$}c<{$}>{$}c<{$}l}
riVti. & \text{varagxyoVga} & \text{aMtara}\\
& 52 & 2 & \text{aMtara}\\
& \text{divxguNisalu} & \text{vagaR} & ~2 & ~2\\
& 104 & 4 & \qq10\; \text{mUla} & 10 & mUla\\
& \text{aMtarada varagxvu} & & \text{kUDi kaLiyalu}&$---$\\
& \quad4 && 12 & 8\\
& \text{kaLiyalu} && \text{adhiRsalu} & $---$\\
& 100 && 6 & 4 & utatxravu.\\
& \text{varagx mUlavu}\\
& 10
\end{tabular}

\begin{center}
{\bf\large 140neV aBayx udAharaNe.}
\end{center}

\begin{enumerate}[\rm (1)]
\item eraDu saMKayxgaLa aMtara $6, 5, 7, 7$ matutx avugaLa varagxyoVga $180, 277, 289, 337$ Agalu, A saMKayxgaLAyxvu?

\item eraDu saMKayxgaLa aMtaragaLu $5\tfrac{1}{2}, 8\tfrac{3}{4}, 7\tfrac{1}{4}$ avugaLa varagx yoVgagaLu $518, 253$ ${9} \above 0pt {16}$, $296$ ${9} \above 0pt {16}$ Adare, A saMKayxgaLAyxvu?
\end{enumerate}

\begin{center}
{\bf 10neV parxkAra.}
\end{center}

udahAraNe, eraDu saMKayxgaLa guNAkAra $42$ avugaLa varagx yoVga $85$ Agalu, A eraDu saMKayxgaLAyxvu.

\begin{center}
{\bf\large sUtarx.}
\end{center}

\begin{verse}
kaM|| rAshi catuguRNadoLagaM | rAshi divxguNavanu varagx yoVgadi kaLidu|| sheVSa saMKayxvanu seVrisi| leVsenisuva varagx mUla kANisimudadiM||

yoVgogaRda saMKayxdavaLa | gAgirutiha rAshi divxguNa kaLiyuta mUlaM||
beVgadi modalina mUlado | LAgaLiyuta kUDu vadhaRgANise samaneY||

vi|| guNAkArada rAshi saMKayxvanunx $4$riMda guNisi, adaralilx A rAshiya divxguNavanunx varagx yoVgadalilx kaLada sheVSavanunx kUDisi adara vagaR mUlavanunx tegadu adaralilx (vagaR yoVgada saMKayxda vaLage rAshiya divxguNavanunx kaLadu sheVda varagx mUlavanunx tegadu) sheVrisi adhiRsidare, oMdu saMKayxvu kaLadu adhiRsidare matotxMdu saMKayxvU baruvadu.
\end{verse}

$
\left.
\begin{tabular}{l>{$}c<{$}>{$}c<{$}}
riVti. & \text{rAshi labadhx} & \text{varagx yoVga}\\
& 42 & 85\\
& 4\; \text{riMda guNisalu} & \text{rAshiya divxguNa}\\
& 168 & 84\\
 & \qq\qquad1\; \text{sheVrisalu} & \text{kaLiyalu}\\[-6pt]
& \quad$---$\\[-6pt]
& \;669 & \qq1\; \text{idanunx}\\
& \text{varagx mUlavu.} & \text{idara varagx mUlavu}\\
& ~13 & 1
\end{tabular}
\right \{
$
\begin{tabular}{>{$}l<{$}>{$}l<{$}}
\text{Agalu} \\
13\; \text{mo, mU} & 13\\
~~1\; \text{eraDaneVmU.} & ~~1\\
\text{kUDi kaLiyalu}& $---$\\
14 & 12\\
\text{adhiRsalu} & $---$\\
~~7 & ~~6\\
\text{utatxragaLu.}
\end{tabular}

\begin{center}
{\bf\large 141neV aBayx udAharaNe.}
\end{center}

\begin{enumerate}[\rm (1)]
\item eraDu saMKayxgaLa guNAkAravu $96, 72, 98, 192,$ matutx avugaLa varagx yoVgagaLu $208, 180, 245, 400$ Agalu, A saMKayxgaLAyxvu?

\item eraDu saMKayxgaLa guNAkAravu $87\tfrac{1}{2},\; 88\tfrac{1}{2},\; 122$ matutx avugaLa varagx yoVgagaLu $205\tfrac{1}{4}, 253\tfrac{9}{16}, 296\tfrac{9}{19}$ Agalu, A saMKayxgaLAyxvu?
\end{enumerate}

\begin{center}
{\bf 11neV parxkAra.}
\end{center}

udAharaNe, eraDu saMKayxgaLa labadhx $5$ avugaLa vagARMtara $96$ Agalu, A eraDu saMKayxgaLAyxvu.

\begin{center}
{\bf\large sUtarx.}
\end{center}

\begin{verse}
kaM|| BAgava rUpaka vagiRsu| tAgadaraMtaradi varagx daMtaraharasi|| beVgadi mUlava tegadada | BAgavu rUpakadiguNise samanaMtakukxM||

$
\left.
\begin{tabular}{ll>{$}c<{$}>{$}c<{$}}
vi|| & BAgalabadhx & \text{rUpasaMKayx} & \text{vagARMtara}\\
& $5$ & 1 & 24)96\\[-6pt]
& vagiRsalu & $---$ & \;$-----$\\
& $25$ & 1 & 4\\
& avugaLa aMtaravu & & \text{varagx mUlavu}\\
& $24$ && 2
\end{tabular}
\right \{
$
\begin{tabular}{>{$}c<{$}>{$}c<{$}}
\text{Agalu}\\
\text{BAga. la.} & \text{rU}\\
5 & 1\\
2 & 2\\
\text{guNisalu} & $---$\\
10 & 2\\
\text{utatxragaLu.}
\end{tabular}
\end{verse}

\begin{center}
{\bf\large 142neV aBayx udAharaNe.}
\end{center}

\begin{enumerate}[\rm (1)]
\item eraDu saMKayxgaLa BAga labadhx $4, 6, 8, 12$ avugaLa vagARMtagaLu $240, 315, 567, 2288$ Agalu, A saMKayxgaLAyxvu?

\item eraDu saMKayxgaLa BAga labadhxgaLu $4\tfrac{1}{4}, 2\tfrac{1}{2}, 3\tfrac{1}{8},$ matutx avugaLa vagARMtaragaLu $153\tfrac{9}{16}, 257\tfrac{1}{4}, 561$ Agalu, A saMKayxgaLAyxvu?
\end{enumerate}

\begin{center}
{\bf\large 12neV parxkAra.}
\end{center}

udAharaNe, eraDu saMKayxgaLa BAga labadhx $9$ matutx avugaLa varagx yoVgavu $328$ Agalu, A eraDu saMKayxgaLAyxvu.

\newpage

\begin{center}
{\bf\large sUtarx}
\end{center}

\begin{verse}
kaM|| BAgava rUpava nogiRsu | tAgavakUDisutaloyxVga saMKayxvanahxrisu||
beVga mUlavanu tegiyuta | BAgava rUpakavaguNise samanaMtakukxM||

vi|| BAga labadhxvanUnx  rUpaveMdare, $1$nunx varigxsi avugaLa yoVgadiMda varagxyoVga saMKayxvanunx \hbox{BAgisi}, A labadhxda mUladiMda BAga labadhxvanunx guNisidare, oMdu saMKayxvu. rUpavanunx guNisidare matotxMdu saMKayU baruvadu.
\end{verse}

$
\left.
\begin{tabular}{l>{$}c<{$}>{$}c<{$}}
riVti. & \text{BAga labadhx} & \text{rUpa}\\
& 9 & 1\\
& \text{varigxsalu} & $---$\\
& 81 & 1\\
& \text{kUDisalu} & $---$\\
& \qq82\; \text{idariMda}& )328\; \text{yoVga saMKayx}\\
&&\qq4 \text{ idara mUlavu } 2
\end{tabular}
\right \{ 
$
\begin{tabular}{>{$}c<{$}>{$}c<{$}}
\text{BAgalabadhx} & \text{rUpa}\\
9 & 1\\
\qq2\; \text{mUla} & \qquad~2\; \text{mU.}\\
\text{guNisalu} & $---$\\
18 & 2\\
\text{utatxravu.}
\end{tabular}

\begin{center}
{\bf\large 143neV aBayx udAharaNe.}
\end{center}

\begin{enumerate}[\rm(1)]
\item eraDu saMKayxgaLa BAga labadhxgaLu $4, 6, 8, 12$ avugaLa varagx yoVgagaLu $272, 333, 585, 2320$ Agalu, A saMKayxgaLAyxvu?

\item eraDu saMKayxgaLa BAga labadhxgaLU $4\tfrac{1}{4}, 2\tfrac{1}{2}, 3\tfrac{1}{8},$ matutx avugaLa varagxyoVgagaLu $171\tfrac{9}{16}, 355\tfrac{1}{4}, 689$ Agalu, A saMKayxgaLAyxvu?
\end{enumerate}

\begin{center}
{\bf 13neV parxkAra.}
\end{center}

udAharaNe, eraDu saMKayxgaLa guNAkAra $36$ avugaLa vagARMtara $65$ Agalu, A saMKayxgaLAyxvu.

\begin{center}
{\bf\large sUtarx.}
\end{center}

\begin{verse}
kaM|| aMtara dogaRdi rAshiya | deMtiralada divxguNisutatx varagxMgoMDu|| aMtu gUDisuta mUlado | LeMtiha daMtaravakUDi kaLadadhiRsutA||

baMtenisida saMKayxgaLi | geMtAguvadadaramUla kANalusamaneY || iMtiha biVjada gaNitava | neMtaMkikxyo LaLiyalaLave matiyalapxnunA\char'366||

vi|| vagARMtarada vagaRdalilx rAshi divxguNada vagaRvanunx kUDisi vagaR mUlavanunx tegadu, A mUladalilx vagARMtara saMKayxvanunx kUDisi adhiRsi, mUlavanunx tegadare, oMdu saMKayxvu kaLadu adhiRsi mUlavanunx tegadare, matotxMdu saMKayxvU baruvadu.
\end{verse}

$
\left.
\begin{tabular}{l>{$}c<{$}>{$}c<{$}}
riVti, & \text{varAgxMtara saMKayx.} & \text{rAshi saMKayx.}\\
& 65 & 36\\
& \text{varigxsalu} & \text{divxguNisalu.}\\
&& 72\\
& 4225 & \text{varigxsalu}\\
&& 5184\\
& \text{A eraDanU kUDisalu}\\
& \qq 9409\\
& \text{varagx mUlavu}\\
& \qq~ 97\\
\end{tabular}
\right \{
$
\begin{tabular}{>{$}c<{$}>{$}c<{$}}
\qq97\; \text{mUla} & \qq97\; \text{mUla}\\
\qq65\; \text{aMtara} & \qq65\; \text{aMtara}\\
\text{kUDi kaLiyalu} & $---$\\
162 & 32\\
\text{AdhiRsalu} & $---$\\
81 & 16\\
\text{mUlavanunx tegiyalu}\\
9 & 4\\
\text{utatxragaLU.}
\end{tabular}

\begin{center}
{\bf\large 144neV aBayx udAharaNe.}
\end{center}

\begin{enumerate}[\rm(1)]
\item eraDu saMKayxgaLa guNAkAra labadhxgaLu $96, 84, 192, 143$ matutx avugaLa varAgxMtatxragaLu $80, 160, 112, 48$ Agalu, A saMKayxgaLAyxvu?

\item eraDu saMKayxgaLa varAgxMtaragaLu $74\tfrac{1}{4}, 86\tfrac{1}{16},$ $67\tfrac{7}{16},$ matutx avugaLa guNAkAra labadhxgaLu $63, 98, 105$ Agalu, A saMKayxgaLAyxvu?
\end{enumerate}

\begin{center}
{\bf 14neV parxkAra.}
\end{center}

heVLidaMthA rAshigaLanunx keVLidaMthA parxmANagaLige sariyAgi viBAgavanunx mADa takakx mAgaRvu.\\

udAhAraNe, $63$nunx $3$kekx $4$ eMba parxmANakekx sariyAda eraDu BAgagaLanunx mADu.

\begin{center}
{\bf\large sUtarx}
\end{center}

\begin{verse}
kaM|| parimANa kANutaliha parisaMKayxgaLoyxVga CeVdasamanaMtakukxM|| irutiha saMKayxgaLaMshake | sarigoLisuta peVLidaSuTx BAgadoLiriyeY||

BAgada parimANagaLira | lAgava samaCeCxVdagoMDu aMshagaLoyxVga|| kAkxgiruvaMshagaLanu bara | dAgiri kuritudanu BAga BAgagaLiMdaM||

vi|| parxmANa saMKayxgaLa yoVgaveV samaCeVdaveMtalU A parimANa saMKayxgaLeV aMshagaLeMtalU tiLadu baradu avugaLiMda parxtayx parxteyxVkavAgi BAgavanunx mADabeVkAda saMKayxvanunx guNisidare baruva labadhxgaLeV viBAga saMKayxgaLAgiruvavu.
\end{verse}

riVti, $3 : 4 = 3 + 4 = 7$ idu sama CeVdavu.
 
\qq\qq Agalu, $\tfrac{3}{7}$ matutx ${4}\above 0pt {7}$ eMba Binanx rAshigaLAdavu.

\qq Adare $9 \times 3 = 27$ idu oMdu BAgavu.
 
\qq\qquad\, $9 \times 4 = 36$ idu oMdu BAgavu.

\begin{center}
{\bf tALe.}
\end{center}

\qq $3 : 4 :: 27 : 36$ parxmANagaLu sariyAgiruvavu. idara vivaravu matutx shidAdhxMtavU saha BUmiti\-sherxVNiyalilx sapxSaTxvAgirutatxde.\\

$(2)$ $42$nunx $\tfrac{1}{2}:\tfrac{2}{3}$ eMba parxmANagaLige sariyAda eraDu BAgagaLanunx mADu.\\

\qq\quad\; $\tfrac{1}{2}:\tfrac{2}{3}=\tfrac{3+4}{6}=\tfrac{7}{6}$ sama CeVdavu.\\

Agalu,\quad~ $\dfrac{\tfrac{1}{2}}{\tfrac{7}{6}}=\tfrac{6}{14}$ Adare $3\times6=18$ idu oMdu BAgavu.\\

matutx, \quad $\dfrac{\tfrac{2}{3}}{\tfrac{7}{6}}=\tfrac{12}{21}$ Adare $2\times 12=24$ idu matotxMdu BAgavu.

\begin{center}
{\bf\large 145neV aBayx udAharaNe.}
\end{center}

\begin{enumerate}[\rm(1)]
\item obabx gaqhasathxnu tananx maganige $1$ rU. $12$ A. Adare, magaLige $1$ rU. $4$ A. barabeVkeMba parxmANadiMda tananx $28000$ rUpAyigaLanunx haMcicx koMDa beVkAdare, yArAyxrigeSeNxSuTx baruvadu?

\item $200$ rUpAyigaLanunx $12$ ANege $2\tfrac{1}{4}$ rUpAyina parxmANadiMda haMci koDu?
\end{enumerate}
