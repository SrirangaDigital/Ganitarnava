\newpage

\begin{center}
\bf{\LARGE{31neV parxkaraNa.}}\\
\vskip .4cm
{\large\bf laGutama sama CeVdavu.}
\end{center}

laGutama samaCeVdaveMdare Binanx rAshigaLa CeVdagaLU aneVka parxkAravAgidAdxgUyx avugaLa bele vetAyxsavAgada hAge elAlxnU samavAda CeVdagaLanunxLaLxvugaLAgi mADuva riVtiyu.

\begin{center}
{\large\bf sUtarx.}
\end{center}

\begin{verse}
kaM|| irutihaCeVdaMgaLigaM| barutiha laGutamApavatayxRvategadada|| nirisi sAmAnayx CeVdake| irutiha CeVdagaLoLahxrisutaMshadi guNisu||

vi|| elAlx CeVdagaLa laGutama sAmAnayx apavatayxRvanunx mADi adanunx aMshagaLigelAlx samavAda CeVdaveMdu tiLadu adanunx AyA CeVdagaLiMda BAgisi, adaradaraMshagaLiMda guNisi bariya beVku.

udAharaNeyu, $\frac{1}{2}, \frac{2}{3}, \frac{3}{4}$ivugaLa sama CeVdagaLanunx mADu.

$2,3,4=12$ idu CeVdagaLa laGutamApavatayxRvu. idu elalxkUkx sama CeVdavAgiruvadu. Agalu.
\begin{equation*}
\left.
\begin{aligned}
& \text{$12\div2=6\times1=6$ idu modalaneVdara aMshavu.}\\
& \text{$12\div3=4\times2=8$ idu eraDaneVdara aMshavu.}\\
& \text{$12\div4=3\times3=9$ idu mUraneVdara aMshavu.}
\end{aligned}
\right\}
\begin{matrix}
\text{ivugaLa keLage A sama}\\
\text{CeVdavanunx baruyalu}
\end{matrix}
\end{equation*}
$\frac{6}{12},\frac{8}{12},\frac{9}{12}$ athavA $\frac{6,8,9}{12}$ I parxkArakekx bariyabahudu.
\end{verse}

idaralilx CeVdagaLAda $2, 3, 4$ eMbavugaLige laGutama sama CeVdavanunx mADalu $12$ baMtu. idu sama CeVdavu. A meVle I $12$nunx modalaneVdara CeVda $2$riMda BAgisi, adanunx adara aMsha $1$riMda guNisalu $6$ Ayitu. idu $\frac{1}{2}$kekx baMda hosa aMshavu. hAge sama CeVda $12$nunx eraDaneVdara CeVda $3$riMda BAgisalu $4$ idanunx adara aMsha $2$riMda guNisalu $8$ idu $\frac{2}{3}$kekx baMda hosa aMshavu.

\vskip .2cm

hAgeV sama CeVda $12$nunx mUraneVdara CeVda $4$riMda BAgisalu $3,$ idanunx adara aMsha $3$riMda guNisalu $9$ idu $\frac{3}{4}$kekx baMda hosa aMshavu. Agalu baMda hosa aMshagaLu $6, 4, 3$ ivugaLigelAlx sama CeVdavAda $12$nunx bariyalu $\frac{6}{12}, \frac{8}{12}, \frac{9}{12}$ eMdu Adavu ivugaLanunx $\frac{6,8,9}{12}$ hiVge bariyuva saMparxdAyavU uMTu.

\eject

\begin{center}
{\bf\Large 38neV aBayx udAharaNe.}
\end{center}

\begin{center}
\begin{tabular}{>{$}c<{$}>{$}c<{$}>{$}c<{$}>{$}c<{$}>{$}c<{$}>{$}c<{$}>{$}c<{$}>{$}c<{$}>{$}c<{$}>{$}c<{$}>{$}c<{$}>{$}c<{$}>{$}c<{$}>{$}c<{$}}

(1)& \frac{5}{6}, & \frac{2}{7}, & \frac{2}{9}, & \frac{11}{12} & & &(2) & \frac{1}{3}, & \frac{2}{9}, & \frac{1}{7}\\ [15pt]

(3) & \frac{2}{3}, & \frac{3}{5}, & \frac{6}{7} & & & & (4) & \frac{2}{11},& \frac{1}{3}, & \frac{2}{9}\\[15pt]

(5) & \frac{7}{10}, & \frac{3}{8} & & & & & (6)& \frac{24}{36}, & \frac{45}{54}, & \frac{12}{24}\\[15pt]

(7) & \frac{9}{7}, & \frac{7}{15} & & & & &(8) & \frac{6}{9}, & \frac{5}{7}\\[15pt]

(9) & \frac{11}{14}, & \frac{16}{25} & & & & & (10) & \frac{10}{15}, & \frac{7}{28}\\[15pt]

(11) & \frac{3}{5}, & \frac{6}{11}, & \frac{5}{17}, & \frac{3}{7} & & & (12) & \frac{3}{7}, & \frac{8}{21}\\[15pt]

(13) & \frac{1}{2}, & \frac{3}{4}, & \frac{5}{6}, & \frac{7}{8}, & \frac{9}{10} & & (14) & \frac{3}{15}, & \frac{2}{6}, & \frac{7}{15}, & \frac{21}{28}, & \frac{11}{24}\\[15pt]

(15) & \frac{3}{4}, & \frac{2}{9}, & \frac{7}{16}, & \frac{9}{25}, & \frac{5}{36} & & (16) & \frac{2}{3}, & \frac{5}{6}, & \frac{4}{9}, & \frac{7}{12}, & \frac{6}{16}\\[15pt]

(17) & \frac{7}{16}, & \frac{11}{18}, & \frac{17}{14}, & \frac{19}{36}, & \frac{25}{48} & & (18) & \frac{2}{3}, & \frac{4}{9}, & \frac{16}{27}, & \frac{8}{81}, & \frac{16}{243}\\[15pt]

(19) & \frac{4}{7}, & \frac{3}{10}, & \frac{5}{12}, & \frac{17}{35}, & \frac{4}{63}, & \frac{15}{28} & (20) & \frac{11}{27}, & \frac{17}{24}, & \frac{5}{6}, & \frac{7}{15}, & \frac{2}{9}, & \frac{35}{36}\\[15pt]
\end{tabular}
\end{center}
